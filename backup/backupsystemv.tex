

Le comportement du programme \texttt{init} est contrôlé par le fichier \texttt{/etc/inittab}, qui organise l'initialisation autour de différents niveaux d'exécution (runlevels). Dans \textbf{KRAKEN OS}, ces niveaux sont utilisés comme suit :

\begin{itemize}
  \item \textbf{0} — arrêt du système (halt)
  \item \textbf{1} — mode mono-utilisateur (Single user mode)
  \item \textbf{2} — niveau définissable par l'utilisateur
  \item \textbf{3} — mode multi-utilisateur complet (sans interface graphique)
  \item \textbf{4} — niveau définissable par l'utilisateur
  \item \textbf{5} — mode multi-utilisateur complet avec gestionnaire d'affichage
  \item \textbf{6} — redémarrage du système (reboot)
\end{itemize}

Le niveau d'exécution par défaut dans KRAKEN est généralement le niveau 3 ou 5, selon que l'on souhaite démarrer avec ou sans interface graphique.

\paragraph{Avantages}

\begin{itemize}
  \item Système établi et bien documenté.
  \item Facile à personnaliser via des scripts shell.
\end{itemize}

\paragraph{Inconvénients}

\begin{itemize}
  \item Temps de démarrage plus long : un système KRAKEN de base démarre en 8 à 12 secondes, mesuré depuis le premier message du noyau jusqu'à l'invite de connexion. La connectivité réseau est généralement établie 2 secondes après l'invite.
  \item Traitement séquentiel des services au démarrage : un retard dans un processus, comme la vérification du système de fichiers, retarde tout le démarrage.
  
  \item Ajout de nouveaux scripts nécessite une planification manuelle et une séquence d'exécution statique.
\end{itemize}
