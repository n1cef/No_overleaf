\chapter*{Résumé}

Ce projet vise à implémenter une nouvelle distribution Linux dédiée aux scientifiques et aux utilisateurs académiques. L’idée est de fournir un environnement ultime, capable d’offrir toutes les configurations et les outils nécessaires pour répondre aux exigences spécifiques de chaque domaine scientifique. Cette distribution permet également de basculer instantanément entre différents environnements (par exemple : de DevOps à un environnement mathématique, puis à un environnement physique, etc.).
\bigbreak
De plus, notre distribution est capable de charger simultanément plusieurs environnements scientifiques sans conflit ni problème de compatibilité.
\bigbreak
Nous avons réalisé des avancées significatives : le système de base a été implémenté, et une image ISO bootable est disponible pour l’utilisateur. Cependant, le projet n’est pas encore totalement achevé. Plusieurs aspects doivent être \textbf{améliorés} dans les prochaines versions : le mécanisme de bascule entre domaines scientifiques, le gestionnaire de paquets, les thèmes, l’ergonomie générale du système, ainsi que la documentation destinée aux utilisateurs.
\bigbreak
La difficulté principale réside dans la construction du système lui-même, qui est une tâche complexe, longue et parfois pénible, ce qui nous a empêchés de finaliser toutes les étapes prévues.

\medskip
\bigbreak
\noindent\rule[2pt]{\textwidth}{0.5pt}

{\textbf{Mots clés :}} GNU/Linux, compilation, gestionnaire de paquets, ISO bootable, système d’exploitation

\noindent\rule[2pt]{\textwidth}{0.5pt}
