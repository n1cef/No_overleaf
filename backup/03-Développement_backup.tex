\chapter{Développement et implémentation}
\clearpage
\setcounter{secnumdepth}{3}

\section{Introduction}
Ce chapitre s'articule autour de cinq phases fondamentales décrivant le processus de conception de notre distribution Linux personnalisée :
\begin{itemize}
    \item Construction du système de base (\textit{Corbuild})
    \item Intégration du noyau Linux et rendu du système amorçable (\textit{Kernel})
    \item Enrichissement par l'ajout de composants complémentaires (\textit{Extendbuild})
    \item Développement du gestionnaire de paquets \textit{Kraken}
    \item Génération de l'image ISO bootable
\end{itemize}

Cette démarche séquentielle permet d'ordonnancer la complexité intrinsèque au développement d'un système d'exploitation autonome tout en justifiant nos orientations technologiques. Chaque étape fait l'objet d'une analyse détaillée mettant en exergue les défis techniques rencontrés et les solutions déployées.

\begin{figure}[htbp]
  \centering
  \includesvg{dev_sections/dev_ressources/dev_process.svg}
  \caption{Processus de développement de KRAKEN-OS}
  \label{fig:dev-process}
\end{figure}


\clearpage


                            %---------------------corebuild------------------------------------------
%---------------------corebuild------------------------
\chapter{ Corebuild : Construction du système de base}
\label{chap:corebuild}
\minitoc
\clearpage

\section{Introduction}
Les chapitres précédents ont posé les bases théoriques et architecturales de notre distribution. Il est désormais temps de passer à la pratique : ce chapitre marque le début du \textbf{développement concret} de Kraken OS. \\
Son objectif principal réside dans l'intégration sélective des composants critiques indispensables au fonctionnement d'un environnement GNU/Linux autonome.



% -----preparation-----------


\section{Préparation de l’environnement hôte}
\label{subsec:env-hote}

Contrairement au développement d’une application (web, mobile ou logicielle) qui s’appuie sur un IDE, un framework ou des bibliothèques, la construction d’un système d’exploitation requiert un environnement hôte complet. Cet environnement doit fournir shell, compilateur, noyau précompilé,
 utilitaires de base  etc...


Il servira de plateforme temporaire pour assembler notre nouveau système, puis pour y faire un \texttt{chroot} une fois les premières étapes terminées.




\subsection{Configuration logicielle}
on utilise un distribution linux minimale nomme \textbf{archlinux} .
Une vérification préalable des outils de compilation s'impose. Les paquets requis incluent notamment :

\begin{itemize}
    \item \textbf{Outils de base} : Bash, Coreutils, Findutils
    \item \textbf{Chaîne de compilation} : GCC, Binutils, Make
    \item \textbf{Utilitaires système} : Gawk, Grep, Sed
    \item \textbf{Gestion d'archives} : Tar, Xz, Gzip
    \item \textbf{Noyau} : Noyau Linux précompilé. 
\end{itemize}

Une attention particulière est portée aux versions logicielles et aux liens symboliques pour garantir la cohérence des dépendances.


\begin{itemize}  
  \item \textbf{Rôle de ces outils} :  
        Ils seront utilisés pour :  
        \begin{itemize}  
          \item Partitionner le disque destiné au système,  
          \item Créer l’arborescence des systèmes de fichiers,  
          \item Développer le cross-compilateur.  
        \end{itemize}  
  \item \textbf{État final} :  
        À ce stade, nous disposons d’un \textbf{environnement hôte configuré de manière complète}, prêt à construire notre système.  
\end{itemize}  
\textcolor{blue}{Pour plus d’informations sur  la préparation de l’environnement hôte  consultez} \cite{lfs_book} 
 \subsection{Partitionnement du disque}
\label{sssec:partitionnement}

Nous disposons d’un disque temporaire dédié à la construction de la distribution. Il sera monté dans un répertoire spécifique et contiendra notre nouvelle distribution jusqu’à la phase de création de l’image ISO .\\
\textcolor{blue}{Pour en savoir plus sur le partitionnement, les types de disques et les schémas de partitions, reportez‑vous au \cite{archlinux_partition}}.\\



Dans \textsc{Kraken OS}, nous avons choisi ce disque temporaire pour l’ensemble du processus de build. Le schéma de partitionnement adopté respecte les standards UEFI modernes :

\begin{itemize}
  \item \texttt{/dev/sda1} : partition racine (ext4, 150 Go) contenant le système de fichiers ;
  \item \texttt{/dev/sda2} : partition \texttt{/home} (ext4, 100 Go) pour les fichiers et répertoires utilisateurs  ;
  \item \texttt{/dev/sda3} : partition EFI (vfat, 1 Go) ;
  \item \texttt{/dev/sda4} : espace de swap (8 Go).
\end{itemize}
\textbf{Outils utilisés} : \texttt{cfdisk}, \texttt{mkfs.ext4}, \texttt{mkfs.vfat -F 32}, \texttt{mkswap}, \texttt{swapon}, \texttt{mount}, \ldots \texttt{umount}.\\
\noindent\textit{Remarque :} ce disque sera monté en \texttt{/mnt/kraken}.
\begin{itemize}
    
\item \textbf{État final} :  
        À ce stade, le disque dur est :  
        \begin{itemize}  
          \item Correctement configuré (partitionnement, formatage),  
          \item Monté dans notre repertoire cible,  
          \item Prêt à héberger l’arborescence du système.  
        \end{itemize}  
\end{itemize}


\clearpage
\subsection{Hiérarchie du système de fichiers }
\label{sssec:hierarchie-preconfig}

Comme discuté au section \textcolor{blue}{ \ref{sssec:fhs}}, nous restons au plus près du standard FHS (Filesystem Hierarchy Standard).

À ce stade, nous devons créer l’arborescence du système de fichiers \textbf{sur le disque monté}. Cette structure doit correspondre à celle illustrée dans la figure ci-dessous : 

\begin{figure}[H]
  \centering
  \includegraphics[width=1\textwidth, height=10cm]{images_pfe/minimalfhs.png}
  \caption{Hiérarchie minimale des répertoires conforme au standard FHS}
  \label{fig:minimlfhs}
\end{figure}

\textbf{Remarque :}  
\begin{itemize}
  \item \textbf{Répertoire \texttt{sources}} :  
        Servira à compiler tous les paquets du système.
  \item \textbf{Répertoire \texttt{tools}} :  
        Contiendra le cross-compilateur et sa toolchain.
\end{itemize}

\subsection{Pré-configuration}
Lorsqu’on est connecté en tant que \texttt{root}, une seule erreur peut endommager ou détruire le système. \\Nous devons donc créer un utilisateur dédié et configurer son groupe ainsi que les droits d’accès sur les répertoires et fichiers de l’arborescence. Cet utilisateur sera utilisé pour compiler et configurer les paquets.

\textbf{Outils utilisés}~:  
\texttt{groupadd}, \texttt{useradd}, \texttt{passwd}, \texttt{chown}, \ldots 

Nous devons également ajouter des configurations supplémentaires relatives aux options de compilation et à l’environnement pour préparer le développement de notre compilateur croisé.\\
\clearpage
Exemple de configuration du \texttt{.bashrc} :

\begin{verbatim}
set +h
umask 022
KRAKEN=/mnt/kraken
KRAKEN_TGT=x86-64-kraken-linux-gnu
PATH=$KRAKEN/tools/bin:$PATH

\end{verbatim}

\textbf{Explications~:}

\begin{itemize}
    \item \textbf{set +h:} 
          Désactive l'utilisation de la table de hachage (\emph{hash table}) de Bash. 
          L idee est de \textbf{FORCER} Le shell de rechercherer  les exécutables dans le \texttt{PATH}   plutôt que de mémoriser leurs emplacements.

    \item \textbf{umask 022:} 
          Définit les permissions par défaut pour les nouveaux fichiers/répertoires~: 
         

    \item \textbf{KRAKEN=/mnt/kraken~:} 
          Variable pointant vers le disque monté où notre système sera construit.

    \item \textbf{KRAKEN-TGT=x86-64-kraken-linux-gnu~:} 
          Définit la cible du compilateur croisé. Cette configuration est importante permet de simuler un environnement hôte différent, comme détaillé dans la section suivante.

   

    \item \textbf{/mnt/kraken/tools/bin:PATH~:}
            Permet au shell de détecter immédiatement notre compilateur croisé situé dans \texttt{/tools/bin}.
\end{itemize}

\textcolor{blue}{Pour plus d’informations sur  la préparation de l’environnement hôte  consultez \cite{lfs_book} }









% ----- Compilation croisée -----------
\section{Construction du compilateur croisé et des bibliothèques associées}
\label{subsec:build-cross}
Les etapes principales sont:\\
\begin{enumerate}
  \item Developemnt du compilateur croisé.
  \item Utilisation de ce compilateur pour assembler une chaîne d’outils temporaire.
  \item Emploi de cette chaîne d’outils pour bâtir le système final.
\end{enumerate}

\subsection{Compilateur croisé}
Comme discuté dans le chapitre  \ref{subsec:cross-compiler} sur la compilation, nous devons créer un compilateur croisé «~factice~».  




\subsubsection{Concept clé~: le triplet système}  
Le système de construction basé sur \texttt{autoconf} utilise un format \texttt{cpu-vendor-kernel-os} (appelé «~triplet système~»).

%\textbf{Remarque~:} Si vous vous interrogez sur l’appellation «~triplet~» pour une structure à quatre composants,  voir la section~\hyperref[sec:triplet]{\textcolor{blue}{\ref*{sec:triplet}}} de l'annexe. 

%\subsubsection{Implémentation du compilateur factice}  
Pour simuler le compilateur croisé~:  
\begin{itemize}  
  \item Il faut Modifiez la cible (\texttt{--target}) durant la compilation des paquets~;  
  \item \textbf{Spécifiquement}~: le champ \texttt{vendor}  du triplet système.  
  \begin{itemize}  
    \item Cette approche indique au compilateur d’installer les paquets dans un emplacement différent (\texttt{/mnt/kraken}) plutôt que sur le système hôte.  
  \end{itemize}  
  \item Ajoutez l’option \texttt{--with-sysroot=/mnt/kraken}~:  
  \begin{itemize}  
    \item Force le compilateur à rechercher les bibliothèques \textbf{uniquement} dans notre système (\texttt{/mnt/kraken}),  
    \item Garantit qu’il ne lie \textbf{aucune bibliothèque} du système hôte.  
  \end{itemize}  
\end{itemize}  


Ce compilateur croisé est composé des paquets suivants~:

\begin{table}[H]
    \centering
    \begin{tabular}{|c|p{8cm}|}
        \hline
        \textbf{Paquet}  & \textbf{Fonction principale} \\
        \hline
       GCC                & Compilateur C/C++ (base de la chaîne d’outils) \\
        \hline
        Binutils             & Fournit l’éditeur de liens (\texttt{ld}) et le vérificateur de dépendances (\texttt{ldd}). \\
        \hline
        Linux-API Headers         &   En-têtes du noyau Linux (nécessaires pour la compilation des bibliothèques système). \\
        \hline
        Glibc          & Bibliothèque C GNU (implémente les fonctions de base comme \texttt{printf}, \texttt{malloc}, etc.) etc.) \\
        \hline
        Libstdc++       & Bibliothèque standard C++ (fournit \texttt{std::string}, \texttt{std::vector}, etc.). \\
       
       
        \hline
    \end{tabular}
    \caption{Composants du compilateur croisé}
    \label{tab:crosscompiler}
\end{table}

%\begin{table}[H]
%    \centering
%    \begin{tabular}{|c|c|p{8cm}|}
%        \hline
%        \textbf{Paquet} & \textbf{Version} & \textbf{Fonction principale} \\
%        \hline
%       GCC             & 14.2.0    & Compilateur C/C++ (base de la chaîne d’outils) \\
%        \hline
%        Binutils        & 2.44       & Fournit l’éditeur de liens (\texttt{ld}) et le vérificateur de dépendances (\texttt{ldd}). \\
%        \hline
%        Linux-API Headers         &6.13.4    &   En-têtes du noyau Linux (nécessaires pour la compilation des bibliothèques système). \\
%        \hline
%        Glibc      & 2.41       & Bibliothèque C GNU (implémente les fonctions de base comme \texttt{printf}, \texttt{malloc}, etc.) etc.) \\
 %       \hline
 %       Libstdc++     & 3.10      & Bibliothèque standard C++ (fournit \texttt{std::string}, \texttt{std::vector}, etc.). \\
       
       
 %       \hline
 %   \end{tabular}
 %   \caption{Composants du compilateur croisé}
 %   \label{tab:crosscompiler}
%\end{table}

Ces paquets doivent être configurés avec un préfixe et une cible spécifiques~:
\begin{verbatim}
--prefix=/mnt/kraken/tools
--with-sysroot=/mnt/kraken
--target=x86_64-kraken-linux-gnu
\end{verbatim}

\noindent
Exemple de configuration du paquet \texttt{gcc}~:

\begin{verbatim}
../configure                  \
    --target=$KRAKEN_TGT      \
    --prefix=$KRAKEN/tools    \
    --with-sysroot=$KRAKEN    \     
    --disable-libstdcxx       \
    --enable-languages=c,c++
\end{verbatim}

\textbf{Explications~:}
\begin{itemize}
  \item \textbf{\texttt{--disable/--enable}}~: 
        Active/désactive des fonctionnalités spécifiques durant la compilation.
        
  \item \textbf{\texttt{--target=x86\_64-kraken-linux-gnu} et \texttt{--with-sysroot=/mnt/kraken}}~: 
        Simulent un compilateur croisé en redirigeant :
        \begin{itemize}
          \item La cible vers une architecture personnalisée,
          \item La recherche des bibliothèques vers le système isolé.
        \end{itemize}
        
  \item \textbf{\texttt{--prefix=/mnt/kraken/tools}}~: 
        Définit le répertoire d’installation du compilateur croisé.
\end{itemize}


\subsection{Construction des outils temporaires}
\label{subsec:outils-temp}

Cette section décrit la compilation des utilitaires de base à l’aide de notre nouvelle chaîne d’outils croisée.

Les paquets suivants sont construits en mode cross-compile :
\begin{table}[H]
    \centering
    \begin{tabular}{|c|p{8cm}|}
        \hline
        \textbf{Paquet} & \textbf{Fonction principale} \\
        \hline
        M4               & Traitement des macros dans le code source \\
        \hline
        Ncurses             & Gestion avancée de l’affichage en terminal \\
        \hline
        Bash              & Interpréteur de commandes principal \\
        \hline
        Coreutils             & Commandes système de base (\texttt{ls}, \texttt{cd}, \texttt{mkdir}, etc.) \\
        \hline
        Diffutils            & Comparaison de fichiers et de répertoires \\
        \hline
        File                & Identification du type de fichiers \\
        \hline
        Grep              & Recherche de motifs dans les fichiers \\
        \hline
        Make               & Automatisation des processus de compilation \\
        \hline
        Gettext              & Utilitaires pour l’internationalisation et la localisation \\
        \hline
        Bison              & Générateur d’analyseurs (parser) \\
        \hline
        Perl              & Langage pratique d’extraction et de rapports \\
        \hline
        Python           & Environnement de développement Python \\
        \hline
        Texinfo             & Programmes pour lire, écrire et convertir des pages \texttt{info} \\
        \hline
        Util-linux        & Divers utilitaires système \\
        \hline
    \end{tabular}
    \caption{Chaîne d’outils essentielle}
    \label{tab:toolchain}
\end{table}

%\begin{table}[H]
 %   \centering
 %   \begin{tabular}{|c|c|p{8cm}|}
 %       \hline
 %       \textbf{Paquet} & \textbf{Version} & \textbf{Fonction principale} \\
 %       \hline
 %       M4             & 1.4.19    & Traitement des macros dans le code source \\
 %       \hline
 %       Ncurses        & 6.5       & Gestion avancée de l’affichage en terminal \\
 %       \hline
 %       Bash           & 5.2.32    & Interpréteur de commandes principal \\
 %       \hline
     %   Coreutils      & 9.5       & Commandes système de base (\texttt{ls}, \texttt{cd}, \texttt{mkdir}, etc.) \\
    %    \hline
    %    Diffutils      & 3.10      & Comparaison de fichiers et de répertoires \\
    %    \hline
    %    File           & 5.45      & Identification du type de fichiers \\
    %    \hline
    %    Grep           & 3.11      & Recherche de motifs dans les fichiers \\
    %    \hline
    %    Make           & 4.4.1     & Automatisation des processus de compilation \\
    %    \hline
    %    Gettext        & 0.24      & Utilitaires pour l’internationalisation et la localisation \\
    %    \hline
    %    Bison          & 3.8.2     & Générateur d’analyseurs (parser) \\
    %    \hline
      %  Perl           & 5.40.1    & Langage pratique d’extraction et de rapports \\
      %  \hline
      %  Python         & 3.13.2    & Environnement de développement Python \\
      %  \hline
      %  Texinfo        & 7.2       & Programmes pour lire, écrire et convertir des pages \texttt{info} \\
      %  \hline
      %  Util-linux     & 2.40.4    & Divers utilitaires système \\
      %  \hline
    %\end{tabular}
    %\caption{Chaîne d’outils essentielle}
    %\label{tab:toolchain}
%\end{table}

Exemple dinstallation du paquete M4 :

\begin{verbatim}
    ./configure --prefix=/usr   \
            --host=x86_64-kraken-linux-gnu \
            
        && 
        make
         &&
        make DESTDIR=/mnt/kraken install    
\end{verbatim}


\section{Installation des logiciels système de base}
\label{subsec:install-base}

À ce stade, nous disposons d’une chaîne d’outils complète et pouvons commencer la construction effective de \textsc{Kraken OS}. Les paquets compilés ici sont les versions finales du système. Avant leur installation, il est nécessaire de se placer dans l’environnement chroot correspondant.




\subsection{Chroot vers le nouveau système}
\label{sssec:chroot}

À ce stade, Nous créons un environnement \texttt{chroot} totalement isolé du système hôte (à l’exception du noyau en cours d’exécution). Pour que cet environnement isolé fonctionne correctement, nous devons monter les systèmes de fichiers virtuels du noyau avant d’y entrer.

\paragraph{Préparation des systèmes de fichiers virtuels}
Dans l’arborescence de montage, nous créons les répertoires suivants :
\begin{verbatim}
/dev
/proc
/sys
/run
\end{verbatim}

\textbf{Explications~:}
\begin{itemize}
  \item \texttt{/proc}~: Interface de communication avec le noyau (processus, paramètres système).
  \item \texttt{/dev}~: Points d’accès aux périphériques matériels.
  \item \texttt{/sys}~: Informations sur le matériel et les pilotes en temps réel.
  \item \texttt{/run}~: Données temporaires volatiles .\\
\end{itemize}
Puis nous montons les systèmes de fichiers du noyau hôte :\\

\begin{verbatim}
mount -vt devpts devpts -o gid=5,mode=0620 /mnt/kraken/dev/pts
mount -vt proc   proc   /mnt/kraken/proc
mount -vt sysfs  sysfs  /mnt/kraken/sys
mount -vt tmpfs  tmpfs  /mnt/kraken/run
\end{verbatim}

\textbf{Explications des montages~:}
\begin{itemize}
  \item \texttt{devpts}~: 
        Système de fichiers des pseudo-terminaux esclaves, essentiel pour les commandes nécessitant un terminal (ex: \texttt{sudo}).
        
  \item \texttt{sysfs}~: 
        Expose les informations matérielles et des pilotes à l’espace utilisateur.
        
  \item \texttt{tmpfs}~: 
        Système de fichiers temporaire en RAM pour les données d’exécution volatiles.
        
  \item \texttt{gid=5, mode=0620}~: 
        \begin{itemize}
          \item \texttt{gid=5}~: Associe le groupe \texttt{tty} (id=5) aux pseudo-terminaux,
          \item \texttt{mode=0620}~: Définit les permissions .
        \end{itemize}
\end{itemize}
\noindent
Une fois ces montages effectués, nous pouvons entrer dans le chroot :

\begin{verbatim}
chroot "/mnt/kraken" /usr/bin/env -i \
  HOME=/root                  \
  PATH=/usr/bin:/usr/sbin     \
  MAKEFLAGS="-j6"      \
  /bin/bash --login
\end{verbatim}
\textbf{Explications des configurations importantes :}
\begin{itemize}
  \item \texttt{chroot "/mnt/kraken" /usr/bin/env -i} :  
        Isole l’environnement en réinitialisant les variables système .



  \item \texttt{MAKEFLAGS="-j6"} :  
        Utilise six cœurs du processeur  pour accélérer la compilation.

  \item \texttt{/bin/bash --login} :  
        Force une session de connexion complète au premier accès du \texttt{chroot}.
\end{itemize}

Avant de débuter l’installation, nous modifions plusieurs configurations critiques dans l’environnement chroot :

\begin{itemize}
  \item \textbf{Création des sous‑répertoires critiques}, par exemple :
    \begin{itemize}
      \item \texttt{/lib/firmware}           : stockage des chargeurs d’amorçage  
      \item \texttt{/etc/sysconfig}          : configurations système  
      \item \texttt{/media/floppy, cdrom}    : points de montage pour supports amovibles  
      \item \texttt{/usr/{,local/}share/\{color,dict,doc,info,locale,man\}} : documentation et données partagées  
      \item \texttt{/var/\{cache,local,log,mail,opt,spool\}}             : fichiers variables  
    \end{itemize}

  \item \textbf{Création de liens symboliques essentiels} :
    \begin{itemize}
      \item \texttt{/proc/self/mounts} → \texttt{/etc/mtab}  
      \item \texttt{/var/run}           → \texttt{/run}  
    \end{itemize}
\end{itemize}









\subsection{Compilation et installation des paquets système finaux avec notre chaîne d’outils}
\label{sssec:install-final}

Nous entrons maintenant dans la phase finale de construction, avec l’installation des 80 paquets fondamentaux :
\begin{table}[H]
\centering
\begin{tabular}{|l|l|}
\hline
\textbf{Catégorie} & \textbf{Exemples de paquets} \\
\hline
Noyau et bas niveau & Glibc, Binutils, GCC \\
\hline
Compression et archivage & Zlib, Xz \\
\hline
Outils de compilation & Make, Autoconf, Meson, Flex \\
\hline
Langages et développement & Python, Perl \\
\hline
Shell et interface utilisateur & Bash, Ncurses, Readline \\
\hline
Utilitaires Unix essentiels & Coreutils, Gawk, Grep \\
\hline
Réseau & IPRoute2, Inetutils \\
\hline
Système de fichiers & Util-linux, Udev \\
\hline
Gestion du système & SysVinit \\
\hline
Documentation & Man-pages, Texinfo \\
\hline
Gestion de paquets & Pkgconf \\
\hline
Chargeur de démarrage & GRUB \\
\hline
\end{tabular}
\caption{Catégories des paquets système finaux}
\label{tab:categories-paquets}
\end{table}



%\begin{itemize}
%    \item \textbf{Noyau et bas niveau} : Glibc-2.40, Binutils-2.43.1, GCC-14.2.0  
%    \item \textbf{Gestion de paquets} : Pkgconf-2.3.0, Dpkg-1.22.6  
%    \item \textbf{Sécurité} : Libcap-2.70, Shadow-4.16.0  
%    \item \textbf{Réseau} : IPRoute2-6.10.0, OpenSSL-3.3.1  
%    \item \textbf{Interface utilisateur} : Ncurses-6.5, Bash-5.2.32  
%\end{itemize}

%Liste complète des paquets critiques :

%\begin{verbatim}
%Man-pages-6.9.1        Iana-Etc-20240806     Glibc-2.40
%Zlib-1.3.1             Bzip2-1.0.8           Xz-5.6.2
%Lz4-1.10.0             Zstd-1.5.6            File-5.45
%Readline-8.2.13        M4-1.4.19             Bc-6.7.6
%Flex-2.6.4             Tcl-8.6.14            Expect-5.45.4
%DejaGNU-1.6.3          Pkgconf-2.3.0         Binutils-2.43.1
%GMP-6.3.0              MPFR-4.2.1            MPC-1.3.1
%Attr-2.5.2             Acl-2.3.2             Libcap-2.70
%Libxcrypt-4.4.36       Shadow-4.16.0         GCC-14.2.0
%Ncurses-6.5            Sed-4.9               Psmisc-23.7
%Gettext-0.22.5         Bison-3.8.2           Grep-3.11
%Bash-5.2.32            Libtool-2.4.7         GDBM-1.24
%Gperf-3.1              Expat-2.6.2           Inetutils-2.5
%Less-661               Perl-5.40.0           XML::Parser-2.47
%Intltool-0.51.0        Autoconf-2.72         Automake-1.17
%OpenSSL-3.3.1          Kmod-33               Libelf (Elfutils-0.191)
%Libffi-3.4.6           Python-3.12.5         Flit-Core-3.9.0
%Wheel-0.44.0           Setuptools-72.2.0     Ninja-1.12.1
%Meson-1.5.1            Coreutils-9.5         Check-0.15.2
%Diffutils-3.10         Gawk-5.3.0            Findutils-4.10.0
%Groff-1.23.0           GRUB-2.12             Gzip-1.13
%IPRoute2-6.10.0        Kbd-2.6.4             Libpipeline-1.5.7
%Make-4.4.1             Patch-2.7.6           Tar-1.35
%Texinfo-7.1            Vim-9.1.0660          MarkupSafe-2.1.5
%Jinja2-3.1.4           Udev                  Man-DB-2.12.1
%Procps-ng-4.0.4        Util-linux-2.40.2     E2fsprogs-1.47.1
%Sysklogd-2.6.1         SysVinit-3.10
%\end{verbatim}


Exemple dinstallation du paquete Flex :

\begin{verbatim}
    ./configure --prefix=/usr \                #Configuration 
            --docdir=/usr/share/doc/flex-2.6.4 \
            
     && 
        make  #Compilation
     &&
        make check   #Test
     &&
         make install  #Installation
     &&
         ln -sv flex   /usr/bin/lex   #Liens symboliques
     &&
         ln -sv flex.1 /usr/share/man/man1/lex.1  #Documentation
\end{verbatim}

\textbf{Remarque :} Ce chapitre est principalement inspiré du livre LFS, que ce soit pour le compilateur croisé, le choix des paquets, leurs versions ou certaines configurations.

\textcolor{blue}{Pour plus d’informations, voir~: \cite{lfs_book}}
 
\section{Conclusion}
À ce stade, nous disposons d’un système minimal fonctionnel. Dans la section suivante, nous verrons la configuration du chargeur d’amorçage, la rédaction des scripts de démarrage et l’intégration du noyau pour rendre le système amorçable.```
\clearpage
\clearpage



                        % ------------------kernel and make the system bootable------------------ ------------------
% kernel and make the system bootable 

\chapter{ Configuration du noyau et amorçage du système et systemV}
\minitoc
\clearpage

\label{sec:kernel-boot}
\section{Introduction}
Démarrer un système Linux implique plusieurs tâches :  
le processus doit monter les systèmes de fichiers virtuels et réels, initialiser les périphériques, vérifier l’intégrité des systèmes de fichiers, monter et activer les partitions ou fichiers d’échange (swap), régler l’horloge système, mettre en service le réseau, démarrer les démons requis et accomplir toute autre tâche personnalisée spécifiée par l’utilisateur.  

\medskip  
Ce processus doit être organisé pour garantir l’exécution des tâches dans le bon ordre et assurer une rapidité optimale.

\section{Configuration des scripts System V}
\label{sssec:sysv-scripts}

System V est le processus d’amorçage classique sous Linux (cf. chapitre \textcolor{blue}{\ref{sssec:sysv}}). Les scripts suivants sont installés dans \texttt{/etc/rc.d} avec un lien symbolique dans \texttt{/etc/init.d}:



\begin{table}[H]
  \centering
  \caption{Exemple Scripts System V et leurs descriptions}
  \label{tab:sysv-scripts}
  \begin{tabularx}{\linewidth}{|l|X|}
    \toprule
    \textbf{Script} & \textbf{Description} \\
    \midrule
    checkfs      & Vérifie l’intégrité des systèmes de fichiers avant leur montage . \\ \hline
    cleanfs      & Supprime les fichiers temporaires entre redémarrages (\texttt{/run}, \texttt{/var/lock}) . \\ \hline
    console      & Charge la table de clavier appropriée et définit la police d’écran. \\ \hline
    halt         & Arrête et éteint le système. \\ \hline
    ifdown       & Désactive une interface réseau. \\ \hline
    ifup         & Active et configure une interface réseau.  \\ \hline
    mountfs      & Monte tous les systèmes de fichiers . \\ \hline
    mountvirtfs  & Monte les systèmes de fichiers virtuels du noyau (\texttt{proc}, \texttt{sysfs}, \texttt{devpts}, \texttt{run}). \\ \hline
    reboot       & Redémarre le système. \\ \hline
    setclock     & Réinitialise l’horloge système à l’heure locale . \\ \hline
    swap         & Active et désactive les partitions ou fichiers d’échange (\texttt{swap}). \\ \hline
     
    \bottomrule
  \end{tabularx}
\end{table}

Tous ces scripts suivent le même modèle : une structure basée sur une simple  `case statement` qui gère les arguments `start`|`stop`|`restart`.  

\begin{figure}[H]  
  \centering  
  \includegraphics[width=1\textwidth]{images_pfe/systemvtemplate.png}  
  \caption{Exemple de template d’un script de démarrage} 
  \label{fig:systemvtemplate}  
\end{figure}  
\textbf{Remarque} : Nous utilisons les scripts de démarrage de BLFS, que nous avons enrichis avec d'autres simple scripts.  
\textcolor{blue}{Pour plus d'informations sur les scripts BLFS, consultez \cite{blfs_bootscripts}.}

%\subsection{Configurations importantes}
%\label{subsec:config-importantes}

%La configuration du système requiert plusieurs fichiers critiques à ajuster :

%\begin{itemize}
 % \item \textbf{Configuration de l’horloge système} — %\texttt{/etc/sysconfig/clock}  
 % Ce fichier définit le comportement de l’horloge (UTC ou heure locale).
 % simplement ellecentien 
 % \begin{verbatim}
 %     UTC=1 #mean set hardware clock  to UTC (Coordinated Universal Time).
 % \end{verbatim}

  %\item \textbf{Configuration de la console Linux} — %\texttt{/etc/sysconfig/console}  
  %Permet de spécifier la table de caractères, la police et la disposition du %clavier.
  % \begin{verbatim}
  %    LOGLEVEL="3"  # Kernel log verbosity level
  %    UNICODE="1"  # Enable Unicode support
  %    FONT="Lat2-Terminus16". # Console font to use
      
  %\end{verbatim}

  %\item \textbf{Configuration de la locale système} et création de %\texttt{/etc/profile}  
  %Le fichier \texttt{/etc/locale.conf} définit la langue et le jeu de %caractères par défaut 

 

  %\item \textbf{Liste des shells valides} — \texttt{/etc/shells}  
  %Contient les chemins des interpréteurs de commandes autorisés pour les %utilisateurs.

  

  %\item \textbf{Configuration réseau} — %\texttt{/etc/sysconfig/ifconfig.enp0s3}, , \texttt{/etc/hosts}  
  %Ces fichiers déterminent l’adresse IP statique, les serveurs DNS et la %résolution locale des noms d’hôtes.

  
%\end{itemize}

\section{Compilation et paramétrage du noyau Linux}
\label{subsec:compilation-noyau}

Notre implémentation s’appuie sur le noyau version \texttt{6.5.10}, retenu pour son support matériel étendu et ses correctifs de sécurité récents. Nous devons configurer, compiler puis installer ce noyau.

Le noyau Linux comporte environ 12 000 options de configuration. Il est essentiel de sélectionner précisément celles dont nous avons besoin pour optimiser la taille, la sécurité et les performances.
\clearpage
La configuration du noyau Linux s'effectue généralement via l'interface menuconfig, accessible grâce à la commande :\\
\begin{verbatim}
   make menuconfig
\end{verbatim}

\begin{figure}[H]
  \centering
  \includegraphics[width=1\textwidth]{images_pfe/kernelmenuconfig.png}
  \caption{Configuration de noyau linux}
  \label{fig:configuration de noyau}
\end{figure}
Nous avons activé environ 1 695 paramètres de configuration.

\textbf{Remarque} : nous n'avons pas configuré manuellement les 1 695 options. Lors de la première étape, nous nous sommes appuyés sur la commande \textbf{make defconfig}, qui génère une configuration de base adaptée à notre architecture \textbf{x86\_64}. 

Ensuite, à l'étape 2, nous avons utilisé \texttt{make menuconfig} pour activer manuellement d'autres options supplémentaires, soit en tant qu’intégrées (\texttt{<*>}), soit en tant que modules (\texttt{<m>}).

Voici un exemple de ces paramètres activés :

\begin{figure}[H]  
  \centering  
  \includegraphics[width=1\textwidth]{images_pfe/numberconfig.png}  
  \caption{Aperçu des paramètres de configuration activés dans \texttt{/boot/config}}  
  \label{fig:kernel_config}  
\end{figure}  



\clearpage
\paragraph*{Fichiers critiques après installation}
\begin{itemize}
  \item \texttt{arch/x86/boot/bzImage} — image binaire du noyau   
  \item \texttt{.config} — fichier de configuration contenant toutes les options sélectionnées  
\end{itemize}

\textit{Remarque :} ces fichiers doivent être configure dans \texttt{/boot} pour être détectés par le chargeur d’amorçage (GRUB) :

\begin{verbatim}
 if faut copier:
 arch/x86/boot/bzImage  dans /boot/vmlinuz-kraken
 config              dans /boot/config
\end{verbatim}


\textcolor{blue}{Pour plus d’informations sur la configuration et la compilation de noyau, consultez \cite{linuxkernel}}
\section{Configuration du processus de démarrage}
\label{sssec:boot}

Le processus de démarrage s’appuie sur deux fichiers principaux : \texttt{/etc/fstab}, qui définit comment et où les périphériques de stockage  sont montés dans la structure de répertoires du système au démarrage, et le fichier de configuration de GRUB \texttt{grub.cfg}.


\begin{enumerate}

  \item \textbf{Exemple de Fichier \texttt{/etc/fstab}}  
  
  \begin{verbatim}


# système de fichiers | point de montage| type | options | dump |fsck-order
#monter les partitions du disque
/dev/sda4      /              ext4     defaults            1     1
/dev/sda3      swap           swap     pri=1               0     0
/dev/sda1      /boot/efi      vfat     codepage=437        0     1
/dev/sda5      /home          ext4     defaults            1     2

#monter les systèmes de fichiers virtuels
proc           /proc          proc     nosuid,noexec,nodev 0     0
sysfs          /sys           sysfs    nosuid,noexec,nodev 0     0
devpts         /dev/pts       devpts   gid=5,mode=620      0     0
tmpfs          /run           tmpfs    defaults            0     0
devtmpfs       /dev           devtmpfs mode=0755,nosuid    0     0
tmpfs          /dev/shm       tmpfs    nosuid,nodev        0     0



  \end{verbatim}
 \textbf{Explications :}
  \begin{itemize}
    \item \textbf{Les quatre premières lignes} servent à monter les partitions du disque dur et activer la swap.
    \item \textbf{Les sept autres lignes} permettent de monter les systèmes de fichiers virtuels.
    \item \textbf{Options :}
    \begin{itemize}
      \item \textbf{dump :} Détermine si une sauvegarde du système de fichiers doit être effectuée (1 = activé, 0 = désactivé). Généralement activé pour les partitions critiques.
      \item \textbf{fsck-order :} Priorité de vérification du système de fichiers au démarrage (0 = aucune, 1 = première priorité, 2 = seconde).\\
    \end{itemize}
  \end{itemize}


  \item \textbf{ Exemple Fichier de configuration de GRUB}  

  \begin{verbatim}

set default=0
set timeout=10
insmod ext2
set root=(hd0,4) 

menuentry "KRAKEN-OS (mode normal)" {
    linux /boot/vmlinuz_kraken root=/dev/sda4 ro
}

menuentry "KRAKEN-OS (mode debug)" {
    linux /boot/vmlinuz_kraken root=/dev/sda4 debug ro
}

menuentry "KRAKEN-OS (mode RAM)" {
    linux /boot/vmlinuz_kraken root=/dev/sda4 ram ro
}

  \end{verbatim}
\end{enumerate}
\textbf{Explications :}
  \begin{itemize}
    \item \textbf{default :} Entrée de menu sélectionnée par défaut (0 = première entrée).
    \item \textbf{timeout :} Délai avant le démarrage automatique (10 secondes).
    \item \textbf{insmod :} Charge un module GRUB nécessaire (ici, le support ext2).
    \item \textbf{menuentry :} Définit une entrée de menu avec des paramètres de démarrage spécifiques.
  \end{itemize}


\clearpage
À ce stade, nous pouvons démarrer le système minimal et visualiser le processus de démarrage.

\begin{figure}[H]
    \centering
    \includegraphics[width=1\textwidth]{images_pfe/corebuildbooscrpts.png}
    \caption{Processus de démarrage}
    \label{fig:bootproc}
\end{figure}

\begin{figure}[H]
    \centering
    \includegraphics[width=1\textwidth]{images_pfe/corebuildresult.png}
    \caption{Interface TTY du système minimal \textsc{Kraken OS}}
    \label{fig:tty}
\end{figure}

\section{Conclustion}
À ce stade, nous disposons d’une distribution Linux minimale et fonctionnelle, capable de démarrer et d’exécuter des commandes de base.\\
Dans le prochain chapitre, nous passerons à l’étape cruciale : transformer cette base en un système complet et polyvalent.
\clearpage



                        %----------------------------extends build------------ ------------------------------------


%----------extends build------------ 
\chapter{  Extendbuild : Ajout des composants avancés}
\minitoc
\clearpage

\section{Introduction}
L’enrichissement du système minimal s’articule autour de deux axes :

\begin{itemize}
 
  \item Installation d’outils et de paquets complémentaires (sécurité, gestion de fichiers et de disques, éditeurs de texte, réseau, etc.).
  \item Configuration de l’environnement graphique ;  
\end{itemize}

\section{Sécurité}
La sécurité se décline en trois volets : l’accès, la prévention et la détection.

\begin{enumerate}
  \item \textbf{Accès} : gestion des sessions utilisateur via l’authentification (login) et renforcement des politiques d’accès à l’aide de modules PAM. Protection de l’accès réseau par des règles iptables (pare‑feu).
  \item \textbf{Prévention} : lutte contre les maliciels (trojans) et chiffrement des données, notamment avec GnuPG.
  \item \textbf{Détection} : surveillance des altérations de fichiers critiques grâce à des outils enregistrant des « empreintes » (hashes).
\end{enumerate}

Exemple des Outils de sécurité installés:
\begin{table}[H]
    \centering
    \begin{tabular}{|c|p{8cm}|}
        \hline
        \textbf{Paquet}  & \textbf{Fonction principale} \\
        \hline
       
        CrackLib  & Vérification de la robustesse des mots de passe contre les attaques par dictionnaire. \\
        \hline
        iptables  & Configuration du pare-feu natif du noyau Linux via des règles de filtrage. \\
        \hline
        GnuPG  & Chiffrement/déchiffrement de données et signatures numériques (implémentation OpenPGP). \\
        \hline
        Sudo  & Délégation sécurisée de commandes privilégiées aux utilisateurs autorisés. \\
        \hline
        Linux-PAM & Framework modulaire pour l'authentification système (login, mot de passe, etc.). \\
        \hline
        OpenSSH  & Suite d'outils de connexion sécurisée à distance . \\
        \hline
        Polkit  & Contrôle granulaire des privilèges pour les processus non-root. \\
        \hline
        Shadow  & Gestion sécurisée des comptes utilisateurs et des mots de passe chiffrés. \\
        \hline
    \end{tabular}
    \caption{Paquets critiques pour la sécurité système}
    \label{tab:security_packages}
\end{table}
 
%\begin{table}[H]
%    \centering
%    \begin{tabular}{|c|c|p{8cm}|}
%        \hline
%        \textbf{Paquet} & \textbf{Version} & \textbf{Fonction principale} \\
%        \hline
%        make-ca & 1.14 & Gestion des certificats racines TLS et mise à jour des magasins de confiance système. \\
 %       \hline
 %       CrackLib & 2.10.2 & Vérification de la robustesse des mots de passe contre les attaques par dictionnaire. \\
 %       \hline
 %       iptables & 1.8.10 & Configuration du pare-feu natif du noyau Linux via des règles de filtrage. \\
 %       \hline
 %       GnuPG & 2.4.5 & Chiffrement/déchiffrement de données et signatures numériques (implémentation OpenPGP). \\
 %       \hline
 %       Sudo & 1.9.15p5 & Délégation sécurisée de commandes privilégiées aux utilisateurs autorisés. \\
%        \hline
 %       Linux-PAM & 1.6.1 & Framework modulaire pour l'authentification système (login, mot de passe, etc.). \\
  %      \hline
   %     OpenSSH & 9.8p1 & Suite d'outils de connexion sécurisée à distance (SSH, SCP, SFTP). \\
   %     \hline
   %     Polkit & 125 & Contrôle granulaire des privilèges pour les processus non-root. \\
   %     \hline
   %     Shadow & 4.16.0 & Gestion sécurisée des comptes utilisateurs et des mots de passe chiffrés. \\
   %     \hline
   % \end{tabular}
   % \caption{Paquets critiques pour la sécurité système}
   % \label{tab:security_packages}
%\end{table}
                   

%\textbf{Exemple d'installation du paquet polkit-125} : outil pour définir et gérer les autorisations.  
%Il permet à des processus non privilégiés de communiquer avec des processus privilégiés.

%\textbf{Dépendances de polkit} :
%\begin{verbatim}
%GLib-2.80.4      duktape-2.7.0       libxslt-1.1.42       Linux-PAM-1.6.1
%elogind-255.5    GTK-Doc-1.34.0      dbusmock-0.32.1 
%docbook-xml-4.5  docbook-xsl-nons-1.79.2
%\end{verbatim}

%\textbf{Configuration requise du noyau} : 
%\begin{verbatim}
%[NAMESPACES], [USER_NS], [AUDIT] (pour l'authentification PAM),  
%[INOTIFY_USER], [TMPFS_POSIX_ACL], [TMPFS], [CRYPTO], [CRYPTO_USER] (pour elogind).
    
%\end{verbatim}

%\textbf{Exemple de compilation de polkit depuis les sources} :
%\begin{verbatim}
%1: Résoudre toutes les dépendances listées ci-dessus
%2: Télécharger l'archive
 %  wget https://github.com/polkit-org/polkit/archive/125/polkit-125.tar.gz
%3: Vérifier l'intégrité de l'archive
 %  md5sum polkit-125.tar.gz  # Doit être 8e9f2377fc7b4010bd29b97d2e288b4f
%4: Extraire l'archive
 %  tar -xvf polkit-125.tar.gz
%5: Accéder au répertoire
%   cd polkit-125
%6: Créer un répertoire de compilation
%   mkdir build && cd build
%7: Configurer (système de build : Meson avec Ninja)
%   meson setup .. \
 %    --prefix=/usr \
 %    --buildtype=release \
 %    -D man=true \
 %    -D session_tracking=elogind \
 %    -D tests=true
%8: Compiler
%   ninja
%9: Tester
%   ninja test
%10: Installer
 %  ninja install
%11: Nettoyer
%   rm -rf ../../polkit-125
%\end{verbatim}

%\textbf{Programmes installés} :  
%pkaction, pkcheck, pkexec, pkttyagent, polkitd

%\textbf{Bibliothèques installées} :  
%/usr/lib/polkit-1/libpolkit-agent-1.so  
%/usr/lib/polkit-1/libpolkit-gobject-1.so

%\textbf{Configuration} :  
%/etc/polkit-1

%\textbf{En-têtes} :  
%/usr/include/polkit-1

%\textbf{Documentation} :  
%/usr/share/gtk-doc/html/polkit-1



\section{Gestion des systèmes de fichiers et gestion des disques}
\label{subsec:fs-disk}

Exmple des paquets  sont installés pour la gestion des systèmes de fichiers et des volumes :
\begin{table}[H]
    \centering
    \begin{tabular}{|c|p{8cm}|}
        \hline
        \textbf{Paquet}  & \textbf{Fonction principale} \\
        \hline
       
        dosfstools  & Utilitaires pour les systèmes de fichiers FAT/FAT32 ) \\
        \hline
        parted  & Partitionnement avancé des disques avec support GPT, MBR et autres tables de partition \\
        \hline
        smartmontools  & Surveillance  des disques durs pour le diagnostic et la prévention des pannes \\
        \hline
       
    \end{tabular}
    \caption{Paquets de gestion des systèmes de fichiers et disques}
    \label{tab:fs-disk}
\end{table}  

%\begin{table}[H]
 %   \centering
 %   \begin{tabular}{|c|c|p{8cm}|}
 %       \hline
 %       \textbf{Paquet} & \textbf{Version} & \textbf{Fonction principale} \\
 %       \hline
       
  %      dosfstools & 4.2 & Utilitaires pour les systèmes de fichiers FAT/FAT32 (mkfs.fat, fsck.fat) \\
   %     \hline
   %     parted & 3.6 & Partitionnement avancé des disques avec support GPT, MBR et autres tables de partition \\
   %     \hline
   %     smartmontools & 7.4 & Surveillance SMART des disques durs pour le diagnostic et la prévention des pannes \\
    %    \hline
       
   % \end{tabular}
   % \caption{Paquets de gestion des systèmes de fichiers et disques}
   % \label{tab:fs-disk}
%\end{table}  


%\textbf{Exemple d'installation du paquet smartmontools-7.4} : 
%Ce paquet contient des utilitaires (\texttt{smartctl}, \texttt{smartd}) pour surveiller les disques via la technologie S.M.A.R.T. intégrée aux %disques modernes (ATA/SCSI).

%\textbf{Dépendances de smartmontools} :
%\begin{verbatim}
%cURL-8.9.1 ou Lynx-2.9.2, Wget-1.24.5 et GnuPG-2.4.5 (pour les disques chiffrés)
%\end{verbatim}

%\textbf{Exemple de compilation depuis les sources} :
%\begin{verbatim}
%1: Télécharger l'archive
 %  wget https://downloads.sourceforge.net/smartmontools/smartmontools-7.4.tar.gz
%2: Vérifier l'intégrité de l'archive
%   md5sum smartmontools-7.4.tar.gz  # Doit être 178d31a6ff5256c093227ab45a3f52aa
%3: Extraire l'archive
%   tar -xvf smartmontools-7.4.tar.gz
%4: Accéder au répertoire
%   cd smartmontools-7.4
%5: Configurer
%   ./configure --prefix=/usr           \
 %             --sysconfdir=/etc       \
 %             --with-initscriptdir=no \
 %             --with-libsystemd=no    \
 %             --docdir=/usr/share/doc/smartmontools-7.4 
%6: Compiler
 %  make
%7: Tester
%   make check 
%8: Installer (en root)
 %  make install
%9: Nettoyer
%   rm -rf ../smartmontools-7.4
%\end{verbatim}

%\textbf{Programmes installés} :  
%\texttt{smartctl}, \texttt{smartd} et \texttt{update-smart-drivedb}

%\textbf{Configuration} :  
%\texttt{/etc/smartd.conf}

%\textbf{Documentation} :  
%\texttt{/usr/share/doc/smartmontools-7.4}

%\textbf{Activation au démarrage} :  
%Un script d'initialisation SystemV doit être créé pour lancer \texttt{smartd}. Voir l'exemple ci-dessous :

%\begin{figure}[H]
%  \centering
%  \includegraphics[width=0.85\textwidth]{images_pfe/smartdboots.png}
%  \caption{Script SystemV pour le démon smartd}
%  \label{fig:smartd_bootscript}
%\end{figure}

\section{Bibliothèques générales}
\label{subsec:general-lib}

Exemple  des bibliothèques utilitaires et de support :

\begin{table}[H]
    \centering
    \begin{tabular}{|c|p{8cm}|}
        \hline
        \textbf{Paquet}  & \textbf{Fonction principale} \\
        \hline
        dbus-glib & Liaison GLib pour D-Bus - Intégration des services de bus de messages dans les applications GLib \\
        \hline
        Fontconfig  & Configuration et personnalisation des polices système  \\
        \hline
        
        Wayland & Protocole de serveur d'affichage moderne (alternative à X11) pour la composition graphique \\
        \hline
        
    \end{tabular}
    \caption{Bibliothèques système et de support général}
    \label{tab:general-lib}
\end{table}




%\section{Bibliothèques graphiques et polices}
%\label{subsec:graphics-fonts}

%Paquets pour le rendu graphique et la gestion des polices :

%\begin{table}[H]
%    \centering
%    \begin{tabular}{|c|p{8cm}|}
%        \hline
%        \textbf{Paquet}  & \textbf{Fonction principale} \\
%        \hline
%        FreeType & Moteur de rendu de polices vectorielles \\
%        \hline
%        Fontconfig  & Configuration et personnalisation des polices système  \\
%        \hline
%        
%        Pixman  & Bibliothèque de manipulation de pixels bas niveau pour les opérations graphiques  \\
%        \hline
 %   \end{tabular}
 %   \caption{Bibliothèques graphiques et gestion des polices}
 %   \label{tab:graphics-fonts}
%\end{table}

%\textbf{Exemple d'installation du paquet harfBuzz-9.0.0} :  
%Le paquet HarfBuzz fournit un moteur de mise en forme de texte pour les polices OpenType.

%\textbf{Dépendances de harfBuzz} :
%\begin{verbatim}
%GLib-2.80.4       texlive-20240312   Graphite2-1.3.14   ICU-75.1
%FreeType-2.13.3 
%\end{verbatim}

%\textbf{Remarque} :  
%Ce paquet présente une \textbf{dépendance circulaire} avec FreeType-2.13.3 (bibliothèque de rendu de polices TrueType).  
%La procédure correcte est :  \\
%1. Compiler FreeType  \\
%2. Compiler HarfBuzz  \\
%3. Recompiler FreeType  \\

%\textbf{Exemple de compilation de harfbuzz depuis les sources} :
%\begin{verbatim}
%1: Résoudre toutes les dépendances listées ci-dessus
%2: Télécharger l'archive
%wget https://github.com/harfbuzz/harfbuzz/releases/download/9.0.0/harfbuzz-9.0.0.tar.xz
%3: Vérifier l'intégrité de l'archive
%   md5sum harfbuzz-9.0.0.tar.xz  # Doit être 0035c129cb1646ab1cff65e5ef7153db
%4: Extraire l'archive
%   tar -xvf harfbuzz-9.0.0.tar.xz
%5: Accéder au répertoire
%   cd harfbuzz-9.0.0
%6: Créer un répertoire de compilation
%   mkdir build && cd build
%7: Configurer (système de build : Meson avec Ninja)
%   meson setup ..             \
%      --prefix=/usr           \
%      --buildtype=release     \
%      -D graphite2=enabled 
%8: Compiler
%   ninja
%9: Tester
%   ninja test
%10: Installer
%   ninja install
%11: Nettoyer
%   rm -rf ../../harfbuzz-9.0.0
%\end{verbatim}

%\textbf{Programmes installés} :  
%hb-info, hb-ot-shape-closure, hb-shape, hb-subset, hb-view

%\textbf{Bibliothèques installées} :  
%\begin{itemize}
%  \item \texttt{libharfbuzz.so}
%  \item \texttt{libharfbuzz-cairo.so}
%  \item \texttt{libharfbuzz-gobject.so}
 % \item \texttt{libharfbuzz-icu.so}
 % \item \texttt{libharfbuzz-subset.so}
%\end{itemize}

%\textbf{Documentation} :  
%\texttt{/usr/share/gtk-doc/html/harfbuzz}
%\section{Paquets et bibliothèques réseau}
%\label{subsec:networking}

\section{Composants pour la connectivité et la gestion réseau} 

\begin{table}[H]
    \centering
    \begin{tabular}{|c|p{8cm}|}
        \hline
        \textbf{Paquet}  & \textbf{Fonction principale} \\
        \hline
        NetworkManager  & Gestion dynamique des connexions réseau (Wi-Fi, Ethernet, VPN)  \\
        \hline
       
        Net-tools  & Collection d'outils réseau  (\texttt{ifconfig}, \texttt{netstat}, \texttt{route}) \\
        \hline
    \end{tabular}
    \caption{Composants pour la connectivité et gestion réseau}
    \label{tab:network}
\end{table}



\section{Environnement graphique XORG}

\begin{figure}[H]
  \centering
  \begin{minipage}[b]{0.30\textwidth}
    \includegraphics[width=\textwidth]{images_pfe/xorg.jpeg}
    \caption{Xorg}
  \end{minipage}\hfill
  \begin{minipage}[b]{0.30\textwidth}
    \includegraphics[width=\textwidth]{images_pfe/wayland.png}
    \caption{Wayland}
  \end{minipage}
  \caption{Xorg et Wayland}
  \label{fig:xorgwayland}
\end{figure}
\FloatBarrier

Il existe des environnements graphiques sous GNU/Linux :

\begin{itemize}
    \item \textbf{Xorg} : ancien, basé sur une architecture \textbf{client-serveur}.
    \item \textbf{Wayland} : a émergé comme une alternative moderne à Xorg. Il prend en charge de nombreuses animations et personnalisations. Cependant, il dépend fortement des capacités 3D du GPU, ce qui peut entraîner des bugs lorsqu’il est utilisé dans une machine virtuelle.
\end{itemize}

C’est pourquoi nous avons choisi d’implémenter \texttt{Xorg} par défaut dans notre distribution, afin de garantir une \textbf{compatibilité maximale} avec tous les environnements.

\subsection*{Exemples d'applications et bibliothèques Xorg}

\begin{table}[H]
    \centering
    \begin{tabular}{|c|p{8cm}|}
        \hline
        \textbf{Paquet}  & \textbf{Fonction principale} \\
        \hline
        xterm & Émulateur de terminal standard pour Xorg \\
        \hline
        xinit  & Utilitaire de lancement du serveur X et session utilisateur \\
        \hline
        libX11  & Bibliothèque cliente principale X11 (gestion fenêtres/événements) \\
        \hline
        xcursor-themes & Collection d'icônes de curseur pour X11 (thèmes par défaut) \\
        \hline
    \end{tabular}
    \caption{Applications et utilitaires d'entrée Xorg}
    \label{tab:xorg-apps}
\end{table}

\textcolor{blue}{Pour plus d’informations sur Xorg, voir \cite{doc_xorg}.}\\
\textcolor{blue}{Pour plus d’informations sur les bibliothèques Xorg, voir \cite{bibliotheques_xorg}.}

\section{Environnement de bureau}
\label{subsec:desktop-env}


Un environnement de bureau offre une interface plus complète au système d’exploitation, avec un ensemble d’utilitaires et d’applications intégrés.

\textbf{KDE} est un environnement de bureau complet, reposant sur le framework \textbf{Qt}, rassemblant de nombreuses applications et une vaste communauté d’utilisateurs.\\
 KDE  se divise en deux blocs principaux :  
\begin{itemize}
  \item \textbf{KDE Frameworks 6} (KF6)   et \textbf{KDE Plasma 6} 
  
\end{itemize}



\subsection{KDE Frameworks 6}
\label{sssec:kf6}

KDE Frameworks 6 est un ensemble de bibliothèques basées sur Qt 6 et QML

Exemple des  bibliothèques


% KDE Frameworks 6
\begin{table}[H]
    \centering
    \begin{tabular}{|c|p{8cm}|}
        \hline
        \textbf{Paquet}  & \textbf{Fonction principale} \\
        \hline
       
        extra-cmake-modules  & Modules CMake supplémentaires pour la construction des logiciels KDE \\
        \hline
        kapidox  & Outil de génération de documentation API pour les frameworks KDE \\
        \hline
        
    \end{tabular}
    \caption{Composants clés de KDE Frameworks 6}\\
    \label{tab:kf6}
\end{table}

\textcolor{blue}{Pour plusieur information sur kde framework  \cite{framework_kde}}.\\

\subsection{Applications KDE}
\label{sssec:kde-apps}

Pour enrichir l’expérience utilisateur, nous installons plusieurs applications KDE ,Exemple :

\begin{table}[H]
    \centering
    \begin{tabular}{|c|p{8cm}|}
        \hline
        \textbf{Paquet} & \textbf{Fonction principale} \\
        \hline
        konsole  & Terminal émulateur avancé avec onglets et profils personnalisables \\
        \hline
        dolphin & Gestionnaire de fichiers phare de KDE avec navigation par onglets \\
        \hline
        plasma-activities  & Gestion des espaces de travail virtuels et suivi des tâches \\
        \hline
        khelpcenter  & Centre d'aide unifié pour la documentation KDE et manuels \\
        \hline
    \end{tabular}
    \caption{Applications principales de l'écosystème KDE}
    \label{tab:kde-apps}
\end{table}

\subsection{KDE Plasma 6}
\label{sssec:plasma6}

KDE Plasma est un ensemble de paquets construits sur KDE Frameworks et QML, constituant l’environnement d’affichage Plasma.

Exemple des Composants Plasma 6


\begin{table}[H]
    \centering
    \begin{tabular}{|c|p{8cm}|}
        \hline
        \textbf{Paquet}  & \textbf{Fonction principale} \\
        \hline
        kscreenlocker  & Système de verrouillage d'écran sécurisé avec intégration PAM \\
        \hline
        plasma-desktop  & Environnement de bureau principal avec panneau et widgets \\
        \hline
        plasma-workspace  & Couche d'intégration pour la gestion des sessions Plasma \\
        \hline
        breeze  & Thème visuel par défaut de Plasma (icônes, décorations de fenêtre) \\
        \hline
    \end{tabular}
    \caption{Composants fondamentaux de KDE Plasma 6}
    \label{tab:plasma6}
\end{table}

\textcolor{blue}{Pour plusieur information sur kde Plasma   \cite{environnement_plasma}}.\\
\section{Thème KDE Plasma et sélecteur de domaine}
\label{sssec:kde-theme-domain}

Nous avons choisi de personnaliser le thème KDE Plasma pour qu’il reflète l’identité de \textsc{Kraken OS}.  
Cette personnalisation inclut :
\begin{itemize}
  \item la modification du panneau par défaut (couleurs, transparences) ;
  \item l’installation de \texttt{cairo‑dock} pour un lanceur plus ergonomique ;
  \item la création d’environnements visuels distincts pour chaque domaine scientifique (web, mobile, IA/ML, cybersécurité, mathématiques, physique, etc.).
\end{itemize}

Nous avons également développé une application \textbf{simple}, reposant sur les activités KDE Plasma, permettant à l’utilisateur de basculer d’un domaine à l’autre.\\

\begin{figure}[H]
  \centering
  \includegraphics[width=1\textwidth]{images_pfe/defaultswitcher.png}
  \caption{Sélecteur de domaine par défaut de \textsc{Kraken}}
  \label{fig:dks}
\end{figure}

\textcolor{blue}{Pour voir une vidéo de prévisualisation de notre environnement KDE Plasma personnalisé, voir \cite{kde_preview}.}\\

\textcolor{blue}{Le code source de l’application qui permet de changer de domaine est disponible dans \cite{switch_domaine}.}\\
\textcolor{blue}{Pour voir une vidéo de prévisualisation de notre application de changement de domaine, voir \cite{kraken_doamin_switcher}.}\\





 \clearpage

\section{Navigateur web}
\label{subsec:web-browser}

Nous avons choisi d’installer le navigateur \textbf{Firefox}, en raison de sa philosophie open source et de sa simplicité.  
Le navigateur Brave, quant à lui, est un paquet volumineux nécessitant environ 30 Go de données pour sa compilation, car il dépend de Chromium.  
%List des depandances de firefox  :
%\begin{verbatim}
%Cbindgen-0.27.0       GTK+-3.24.43          libnotify-0.8.3   LLVM-18.1.7             
%nodejs-20.16.0        PulseAudio-17.0       Python-3.12.5     startup-notification-0.12
%UnZip-6.0             alsa-lib-1.2.12       SQLite-3.46.1     NASM-2.16.03    
%cURL-8.9.1            Doxygen-1.12.0        FFmpeg-7.0.2      GeoClue-2.7.1            
%liboauth-1.0.3        pciutils-3.13.0       Valgrind-3.23.0    Wget-1.24.5              
%Wireless-Tools-29     yasm-1.3.0            nss-3.103            
%\end{verbatim}
\begin{figure}[H]
  \centering
  \includegraphics[width=1\textwidth]{images_pfe/firefox.png}
  \caption{Navigateur web firefox}
  \label{fig:firefox-custom}
\end{figure}



 
\section{Gestionnaire d’affichage et thème du bootloader GRUB}
\label{subsec:sddm-theme}

\subsection{Thème SDDM}
Un gestionnaire d'affichage (login manager) est une interface graphique lancée au démarrage du système, remplaçant le shell par défaut. 

Pour sa légèreté et sa facilité de personnalisation, nous avons retenu \textbf{SDDM} comme gestionnaire d’affichage.  
Basé sur Qt/QML, il permet une création intuitive de thèmes visuels grâce à son langage déclaratif.

\begin{figure}[H]
  \centering
  \includegraphics[width=1\textwidth]{images_pfe/sddmkrakentheme.png}
  \caption{Thème SDDM personnalisé pour \textsc{Kraken OS}}
  \label{fig:sddm-custom}
\end{figure}

    \textbf{Remarque} : Ce thème, nommé SDDM-Astronaut, est principalement développé par Keyitdev.
Étant donné que le thème est sous licence GPL, nous pouvons le modifier. Nous avons donc apporté des modifications simples\\

\textcolor{blue}{Version originale du thème Astronaut : disponible dans \cite{sddm_theme_astronaut}}.\\
\textcolor{blue}{Notre Version modifiée : disponible dans \cite{sddm_theme_kraken}}.\\
\subsection{Thème GRUB}
\label{subsec:grub-theme}

Le thème GRUB a été adapté à la charte graphique de la distribution. Sa structure minimale comprend :
\begin{itemize}
    \item Un fichier principal \texttt{theme.txt} 
    \item Configuration des polices, couleurs et résolution
   
\end{itemize}



\begin{figure}[H]
  \centering
  \includegraphics[width=1\textwidth]{images_pfe/grubthemekraken.png}
  \caption{Thème GRUB personnalisé pour \textsc{Kraken OS}}
  \label{fig:grub-custom}
\end{figure}

\textcolor{blue}{Code source disponible dans \cite{theme_grub}}.
\section{Conclusion}
À ce stade, nous disposons d'une distribution Linux fonctionnelle et enrichie. Cependant, il manque encore un composant essentiel : un \textbf{gestionnaire de paquets} permettant aux utilisateurs d'installer, mettre à jour ou supprimer des logiciels de manière cohérente.

Dans le prochain chapitre, nous aborderons le développement et l'intégration de notre gestionnaire de paquets 
\clearpage
                                            
                                            
                        %----------------------------------package manager---------------------------------------------
\chapter{ Gestionnaire de paquets : \textsc{Kraken}s}
\minitoc
\clearpage

\label{sec:pkgmgr}

\section{Introduction}
\label{sec:intro}

Pour résoudre les limitations de compilation , nous avons mis en place un ensemble d'outils, de scripts et de mécanismes dédiés à la gestion des paquets.  
Cela inclut :
\begin{itemize}
    \item Une structure de dépôt simple ;
    \item Un format de fichier \texttt{PKGBUILD.kraken} pour décrire les paquets ;
    \item Des outils CLI personnalisés pour installer, mettre à jour, supprimer ou rechercher des paquets ;
    \item Un système de gestion des dépendances ;
    \item Un mécanisme de suivi des métadonnées des paquets ;
\end{itemize}

\subsubsection*{Langages s utilisés:}
\begin{figure}[hbt!]
  \centering
  \begin{minipage}[b]{0.18\textwidth}
    \includegraphics[width=\textwidth]{images_pfe/c.png}
    \caption{C}
  \end{minipage}\hfill
  \begin{minipage}[b]{0.18\textwidth}
    \includegraphics[width=\textwidth]{images_pfe/bash.jpeg}
    \caption{BASH SCRIPT}
  \end{minipage}\hfill
  \begin{minipage}[b]{0.18\textwidth}
    \includegraphics[width=\textwidth]{images_pfe/yml.png}
    \caption{YAML}
  \end{minipage}\hfill
  \begin{minipage}[b]{0.18\textwidth}
    \includegraphics[width=\textwidth]{images_pfe/sqlite.jpeg}
    \caption{SQLITE3}
  \end{minipage}\hfill
  
  \caption{Stack technique de la gestionnaire des paquetes}
  \label{fig:KPGlanguage}
\end{figure}
\FloatBarrier












Notre gestionnaire de paquets se compose de deux composants principaux :
\begin{enumerate}
    \item Le dépôt où sont stockées les métadonnées des paquets ;
    \item Les outils du gestionnaire interagissant avec ce dépôt.
\end{enumerate}

\section{Le dépôt \textsc{Kraken}}
\label{subsec:depot-kraken}

Le dépôt \textsc{Kraken} contient les métadonnées de tous les paquets disponibles, stockées dans des fichiers \texttt{PKGBUILD.kraken}. Chaque fichier décrit la procédure de compilation d'un paquet à partir des sources.

\subsection{Structure du dépôt }
\label{subsubsec:structure-depot}

La structure du dépôt \textsc{Kraken} est conçue pour être lisible, modulaire et maintenable. Chaque paquet correspond à un dossier contenant un fichier \texttt{PKGBUILD.kraken}.\\



\begin{figure}[H]
  \centering
  \includegraphics[width=1\textwidth]{images_pfe/repodotoff.png}
  \caption{Structure du dépôt du gestionnaire de paquets \textsc{Kraken}}
  \label{fig:krakenrepo}
\end{figure}

Le dépôt comporte quatre catégories principales :
\begin{itemize}
    \item \textbf{core} : Contient les paquets fondamentaux (gcc, binutils, coreutils...) ;
    \item \textbf{extend} : Paquets d'extension pour enrichir le système (xorg, fakroot, strace...) ;
    \item \textbf{sc} : Outils pour l'informatique (applications web, outils mobiles...) ;
    \item \textbf{temps\_tools} : Outils temporaires pour la chaîne de compilation croisée.
\end{itemize}

\textbf{Note} : Des autre catégories comme  \textbf{math} et \textbf{phys} existent également mais ont été omises pour simplifier la figure.

Chaque paquet dans ces répertoires est décrit par un fichier \texttt{PKGBUILD.kraken} spécifiant sa procédure de construction.

\subsection{Le fichier \texttt{PKGBUILD.kraken}}
\label{subsubsec:pkgbuild-file}

Le fichier \texttt{PKGBUILD.kraken} est un modèle déclaratif décrivant :
\begin{itemize}
    \item La méthode de construction du paquet ;
    \item Ses dépendances ;
    \item Son processus d'installation.
\end{itemize}

\begin{figure}[H]
  \centering
  \includegraphics[width=1\textwidth, height=10cm]{images_pfe/pkgbuildkrakenmodle.png}
  \caption{Modèle générique de \texttt{PKGBUILD.kraken}}
  \label{fig:pkgtemplate}
\end{figure}
\clearpage
Exemple concret de \texttt{PKGBUILD.KRAKEN} pour le paquette \texttt{gcc } :
\begin{figure}[H]
  \centering
  \includegraphics[width=1\textwidth, height=10cm]{images_pfe/exemplepgkbuldgcc.png}
  \caption{Exemple de \texttt{PKGBUILD.kraken} pour le paquet \texttt{gcc}}
  \label{fig:pkgbuildgcc}
\end{figure}

Principales sections du fichier :
\begin{itemize}
    \item \textbf{pkgname}, \textbf{pkgver} : Nom et Version  du paquet ;
    \item \textbf{dependencies} : Dépendances requises avant installation ;
    \item \textbf{sources} : lien de téléchargement de l'archive ;
    \item \textbf{md5sums} : Vérification d'intégrité de l'archive ;
    \item \textbf{kraken-prepare()} : Préparation (extraction, configuration) ;
    \item \textbf{kraken-build()} : Compilation du paquet ;
    \item \textbf{kraken-test} : Procédure de test ;
    \item \textbf{kraken-preinstall} : Préconfigurations avant installation ;
    \item \textbf{kraken-install()} : Installation des fichiers système ;
    \item \textbf{kraken-postinstall} : Postconfigurations (documentation, fichiers) ;
    \item \textbf{kraken-remove} : Procédure de désinstallation.
\end{itemize}

\subsection{Le fichier pkgindex.kraken}
\label{subsubsec:pkgindex}
Le fichier \texttt{pkgindex.kraken} est un index centralisé au format YAML qui contient les métadonnées essentielles de tous les paquets disponibles.
\begin{figure}[H]
  \centering
  \includegraphics[width=1\textwidth]{images_pfe/pkgindekrakenpkgindex.png}
  \caption{Exemple de fichier pkgindex.kraken}
  \label{fig:pkgindex-example}
\end{figure}

Chaque entrée de paquet contient les champs suivants :
\begin{itemize}
    \item \textbf{category} : Classification fonctionnelle (core, math, physics, etc.)
    \item \textbf{version} : Version de paquets
    \item \textbf{path} : Chemin relatif vers le fichier \texttt{pkgbuild.kraken} dans le depot.
    \item \textbf{dependencies} : Liste des dépendances obligatoires de ce paquetes
    \item \textbf{checksum} : Vérification d'intégrité  du fichier de \texttt{pkgbuild.kraken} de cette paquetes
\end{itemize}
Ce fichier \textbf{généré automatiquement} optimise la recherche de paquets via :
\begin{itemize}
\item \textbf{Réduction} de la complexité de recherche des paquetes dans le depot  de O(n) à O(1) ;
\item \textbf{Mise à jour} instantanée lors des modifications du dépôt ;
\item \textbf{Accès} hors ligne aux métadonnées (versions, dépendances).
\end{itemize}





\clearpage
\subsection{Résumé de la structure du dépôt}
\label{subsec:resume-depot}

Pour résumer cette section :

\begin{itemize}
    \item Le dépôt est organisé en \textbf{répertoires thématiques} classés par domaine (core, extends, sc, math , phys,etc.) ;
    \item Chaque répertoire contient des paquets spécifiques à son domaine ;
    \item Chaque paquet est décrit par un fichier \texttt{PKGBUILD.kraken} qui :
    \begin{itemize}
        \item Est utilise par le gestionnaire de paquets
        \item Contient les instructions de compilation/installation de cette paquette
    \end{itemize}
    \item Le fichier \texttt{pkgindex.kraken} centralise les métadonnées :
    \begin{itemize}
        \item Noms et versions des paquets
        \item Dépendances entre paquets
       \item \textbf{Liens directs} vers les fichiers pkgbuild.kraken de chaque paquet qui existe dans le dépôt.

        \item Permet une recherche optimisée (complexité O(1)) dans le depot.
    \end{itemize}
\end{itemize}





\textcolor{blue}{Code source de la depot (KUR:kraken users repository) disponible dans  \cite{depot_kur}}.
\section{Les outils \textsc{Kraken}}
\label{subsec:kraken-tools}


\




%-------- kraken tools ------------------







Dans cette section, nous présentons les outils en ligne de commande développés pour gérer les paquets à travers le gestionnaire \textsc{Kraken}. Chaque outil est conçu pour exécuter une étape spécifique du cycle de vie d’un paquet : téléchargement, préparation, compilation, test, etc.


\subsection{kraken help}
\label{subsec:kraken-help}

Cet outil fournit à l'utilisateur un menu d'aide comprenant :
\begin{itemize}
    \item La liste des commandes disponibles
    \item Les paramètres acceptés par chaque commande
    \item Des exemples de workflows d'utilisation

\end{itemize}

\begin{figure}[H]
  \centering
  \includegraphics[width=1\textwidth, height=12cm]{images_pfe/krakenhelpmenu.png}
  \caption{Menu d'aide du gestionnaire de paquets \textsc{Kraken}}
  \label{fig:kraken-help}
\end{figure} 

\subsection{kraken download}

\textbf{Utilisation :} \texttt{sudo kraken download <nom\_du\_paquet>}

Cet outil permet de télécharger l’archive (tarball) d’un paquet et de vérifier son intégrité (via \texttt{md5sum}).

\paragraph{Étapes de fonctionnement :}
\begin{enumerate}
    \item Télécharger ou mettre à jour le fichier \texttt{pkgindex.kraken}, et le stocker dans .cache/kraken/krakenindex
    \item Vérifier que le nom du paquet fourni par l'utilisateur existe dans le dépôt depuit le fichier pkgindex.kraken.
    \item Si le paquet existe, créer un répertoire de compilation nommé \texttt{source/}, puis créer un sous-dossier nommé selon le paquet. Dans ce dossier, télécharger le fichier \texttt{PKGBUILD.kraken} et télécharger l’archive du paquet via le champ sources de la ficher pkgbuild.kraken.
    \item Vérifier l'intégrité du fichier \texttt{PKGBUILD.kraken} ainsi que celle de l’archive téléchargée.
    \item Mettre à jour la base de données \texttt{/var/lib/kraken/kraken.db} en marquant le paquet comme téléchargé.
\end{enumerate}

%\begin{figure}[H]
 % \centering
  %\includegraphics[width=1\textwidth, height=10cm]{images_pfe/downlad.png}
  %\caption{Exemple pratique de l’outil \texttt{kraken download}}
  %\label{fig:pkgtemplate}
%\end{figure}




\subsection{kraken prepare}

\textbf{Utilisation :} \texttt{sudo kraken prepare <nom\_du\_paquet>}

Cet outil est conçu pour préparer le paquet avant sa compilation.

\paragraph{Étapes de fonctionnement :}
\begin{enumerate}
    \item Interroger la base de données pour vérifier si le paquet a été téléchargé.
    \item Si oui, localiser l’archive et l’extraire (en utilisant \texttt{tar}, \texttt{unzip}, etc., selon le format).
    \item Extraire la fonction \texttt{kraken\_prepare()} du fichier \texttt{PKGBUILD.kraken} du paquet, et l’exécuter.
    \item Mettre à jour la base de données en marquant le paquet comme préparé (\texttt{prepared = 1}).
\end{enumerate}

%\begin{figure}[H]
 % \centering
  %\includegraphics[width=1\textwidth, height=10cm]{images_pfe/prepare.png}
  %\caption{Exemple pratique de l’outil \texttt{kraken prepare}}
  %\label{fig:pkgtemplate}
%\end{figure}





\subsection{kraken build}

\textbf{Utilisation :} \texttt{sudo kraken build <nom\_du\_paquet>}

Cet outil est conçu pour compiler les paquets à l’aide de leur système de build (\texttt{make}, \texttt{ninja}, \texttt{meson}, etc.).

\paragraph{Étapes de fonctionnement :}
\begin{enumerate}
    \item Interroger la base de données pour vérifier si le paquet a été préparé.
    \item Si oui,  Extraire la fonction \texttt{build()} du fichier \texttt{PKGBUILD.kraken} et l’exécuter pour compiler le paquet (génération de bibliothèques, binaires, etc.).
    \item Mettre à jour la base de données en marquant le paquet comme compilé (\texttt{built = 1}).
\end{enumerate}







\subsection{kraken test}

\textbf{Utilisation :} \texttt{sudo kraken test <nom\_du\_paquet>}

\paragraph{Étapes de fonctionnement :}
\begin{enumerate}
    \item Interroger la base de données pour vérifier si le paquet a été compilé.
    \item Si oui, Extraire la fonction \texttt{test()} depuis le fichier \texttt{PKGBUILD.kraken}, et l’exécuter afin de tester le paquet compilé.
\end{enumerate}

\textbf{Remarque :} les tests sont optionnels. Certains paquets ne fournissent pas de suite de test officielle, donc l'exécution de cette étape n'est pas considérée comme critique.

%\begin{figure}[H]
 % \centering
 % \includegraphics[width=1\textwidth, height=10cm]{images_pfe/test.png}
 % \caption{Exemple pratique de l’outil \texttt{kraken test}}
 % \label{fig:pkgtemplate}
%\end{figure}










\subsection*{Principe de la « fausse installation »}  
\label{subsecc:fakeinstall}
Comme nous l'avons vu dans le chapitre \ref{subsec:suivi-fichiers}, nous devons implémenter un mécanisme qui installe les paquets de manière fictive dans un répertoire temporaire afin de détecter tous les fichiers et répertoires créés par le paquet lors de l'installation. Cette étape sera utilisée ultérieurement dans le mécanisme de suppression des paquets.

\textbf{Utilisation :} \texttt{sudo kraken fakeinstall <nom\_du\_paquet>}

\textbf{Étapes :}
\begin{enumerate}
  \item Vérifier si le paquet a été compiler  ou non à partir de la base de données.
  \item Si oui, Extraire la fonction \texttt{kraken-install} du fichier \texttt{pkgbuild.kraken} et créer un répertoire temporaire situé dans \texttt{/tmp} où le paquet sera installé \textbf{fictivement}, en forçant l’installation dans ce répertoire de destination.
  \item Créer un environnement fictif à l’aide de l’outil \textbf{fakeroot}.
  \item Lancer l’installation du paquet  en parallèle avec l’outil \textbf{strace} afin de détecter tous les changements effectués sur le système.
  \item Générer les métadonnées dans :
  \begin{itemize}
    \item \texttt{/var/lib/kraken/packages/DIRS} -- contient tous les répertoires marqués par le paquet.
    \item \texttt{/var/lib/kraken/packages/FILES} -- contient tous les fichiers marqués par le paquet.
  \end{itemize}
\end{enumerate}

%La figure suivante montre un exemple pratique de l'utilisation de l’outil \texttt{fakeinstall} :

%\begin{figure}[H]
%  \centering
 % \includegraphics[width=1\textwidth, height=10cm]{images_pfe/pkgbuildkrakenmodle.png}
  %\caption{Modèle de fichier \texttt{pkgbuild.kraken}}
  %\label{fig:pkgtemplate}
%\end{figure}


%\textcolor{red}{Cette figure illustre un exemple des fichiers et répertoires de métadonnées générés pour un paquet (\texttt{DIRS} et %\texttt{FILES}) :}

%\begin{figure}[H]
%  \centering
%  \includegraphics[width=1\textwidth, height=10cm]{images_pfe/pkgbuildkrakenmodle.png}
%  \caption{Modèle de fichier \texttt{pkgbuild.kraken}}
%  \label{fig:pkgtemplate}
%\end{figure}

%\textbf{Remarque :} Comme vous pouvez le voir dans la figure, les fichiers \texttt{DIRS} et \texttt{FILES} contiennent certains répertoires %critiques tels que \texttt{/usr}, \texttt{/boot}, \texttt{/bin}, \texttt{/root}. Il est donc nécessaire de mettre en place un mécanisme de %filtrage de ces répertoires afin de garantir que le système de fichiers critique ne soit pas altéré lors de la suppression d’un paquet.





\subsection{kraken install}

\textbf{Utilisation :} \texttt{sudo kraken install <nom\_du\_paquet>}

Cet outil est conçu pour installer les paquets dans le système final.

\textbf{Étapes :}
\begin{enumerate}
  \item Vérifier si le paquet a été installé fictivement (fakeinstall) dans la base de données.
  \item Si oui, Exécuter la fonction \texttt{install} située dans le fichier \texttt{pkgbuild.kraken}.
  \item Marquer le paquet comme installé définitivement dans le système avec la date d’installation.
\end{enumerate}



%\begin{figure}[H]
%  \centering
%  \includegraphics[width=1\textwidth, height=10cm]{images_pfe/install.png}
%  \caption{Modèle de fichier \texttt{pkgbuild.kraken}}
%  \label{fig:pkgtemplate}
%\end{figure}





\subsection{kraken postinstall}

\textbf{Utilisation :} \texttt{sudo kraken postinstall <nom\_du\_paquet>}

Cet outil est conçu pour exécuter les configurations post-installation après l’installation du paquet, comme la modification de certains fichiers de configuration.

\textbf{Étapes :}
\begin{enumerate}
  \item Vérifier si le paquet est installé dans le système en consultant la base de données.
  \item Si oui, Exécuter la fonction \texttt{postinstall} à partir du fichier \texttt{pkgbuild.kraken}.
\end{enumerate}

%La figure suivante montre un exemple d’utilisation de l’outil \texttt{postinstall} :

%\begin{figure}[H]
%  \centering
 % \includegraphics[width=1\textwidth, height=10cm]{images_pfe/alpine.jpg}
 % \caption{Modèle de fichier \texttt{pkgbuild.kraken}}
 % \label{fig:pkgtemplate}
%\end{figure}
%\textcolor{blue}{Pour voir une vidéo de démonstration de l’utilisation de ces outils pour installer un paquet, voir \cite{kraken_tools}.}

\subsection{kraken remove}

\textbf{Utilisation :} \texttt{sudo kraken remove <nom\_du\_paquet>}

Ce mécanisme prend en charge deux cas :

\begin{enumerate}
  \item Détecter si le paquet possède un mécanisme de désinstallation dans son système de construction. Si oui, nous l’utilisons pour supprimer le paquet.
  \item Si le paquet ne possède pas de mécanisme de désinstallation, nous appelons un mécanisme nommé \texttt{manual\_install}. Celui-ci parcourt les fichiers de métadonnées (\texttt{FILES}, \texttt{DIRS}) créés lors de l’étape \texttt{fakeinstall}, et supprime tous les fichiers et répertoires listés.
  \item Mettre à jour la base de données en supprimant le paquet de la table des paquets installés.
\end{enumerate}



%\begin{figure}[H]
 % \centering
  %\includegraphics[width=1\textwidth, height=10cm]{images_pfe/pkgbuildkrakenmodle.png}
  %\caption{Modèle de fichier \texttt{pkgbuild.kraken}}
  %\label{fig:pkgtemplate}
%\end{figure}

\subsection{kraken entropy}

\textbf{Utilisation :} \texttt{sudo kraken entropy <nom\_du\_paquet>}

Cet outil est conçu pour installer \textbf{automatiquement} le paquet ainsi que toutes ses dépendances.

\textbf{Étapes :}
\begin{enumerate}
  \item Construire le graphe complet du paquet, affichant toutes les dépendances et les dépendances des dépendances.
  \item Détecter s'il y a des cycles dans le graphe ou des dépendances de type config ou diamant. Pour en savoir plus sur ces problèmes, référez-vous au section \textcolor{blue}{\ref{subsec:circular-dependencies}}.
  \item Parcourir le graphe en utilisant la recherche en profondeur (Depth First Search) et installer chaque paquet dans le bon ordre en utilisant les outils précédents (\texttt{download}, \texttt{prepare}, \texttt{build}, \texttt{test}, \texttt{fakeinstall}, \texttt{install}, \texttt{postinstall}).
\end{enumerate}

%Exemple illustrant le mécanisme d'entropie :

%\begin{figure}[H]
 % \centering
 % \includegraphics[width=1\textwidth, height=13cm]{images_pfe/phpgraph.png}
 % \caption{Modèle de fichier \texttt{pkgbuild.kraken}}
 % \label{fig:pkgtemplate}
%\end{figure}


\subsection{kraken getversion}

\textbf{Utilisation :} \texttt{sudo kraken getversion <nom\_du\_paquet>}

Cet outil est conçu pour afficher la version d'un paquet, qu'il soit installé ou non ( depuis le dépôt).


\subsection{kraken dependency}

\textbf{Utilisation :} \texttt{sudo kraken getdeps <nom\_du\_paquet> <version\_du\_paquet>}

Généralement, si vous choisissez de construire le paquet manuellement, vous devez vérifier ses dépendances. Cet outil affiche les dépendances d'un paquet en utilisant \texttt{yq} pour interroger le \texttt{pkgindex.kraken}.



\subsection{kraken checkinstaller}

\textbf{Utilisation :} \texttt{sudo kraken checkinstalled <nom\_du\_paquet>}

Cet outil détecte si le paquet est installé dans le système ou non en interrogeant la base de données.

 
\subsection{Outils en développement}

Comme vous pouvez le voir, notre gestionnaire de paquets construit le paquet à partir des sources, mais nous prévoyons d'améliorer cette fonctionnalité dans les futures versions. Nous préparons le développement d'un mécanisme pour :\\

  permettant de compiler le paquet à l'intérieur de Kraken OS, hébergé dans une machine virtuelle sur un serveur cloud. L'idée est de générer un nouveau tarball \textbf{reconstruit} nommé \texttt{pkgnam-pkgversion.kraken}.\\
Ce tarball sera compilé et configuré dans le cloud, et l'utilisateur n'aura qu'à l'installer sur son système sans avoir à le compiler lui-même.
Cela permettra de réduire le temps de compilation, car la construction de paquets à partir des sources est fastidieuse et consomme beaucoup de ressources et de temps. Par exemple, compiler un paquet comme \texttt{gcc} peut prendre 4 heures avec 6 cœurs.\\

\bigbreak

\textcolor{blue}{Code source  de notre gestionnaires de paquetes kraken  disponible dans  \cite{gestionnaire_paquets}}.\\
\bigbreak
\textcolor{blue}{Pour voir une vidéo de démonstration de l’utilisation de ces outils pour installer un paquet manuellement, voir \cite{kraken_tools}.}\\
\bigbreak
\textcolor{blue}{Pour voir une vidéo de démonstration de l’utilisation de loutil, \texttt{kraken-entropy}, pour installer \textbf{automatiquement} un paquet avec \textbf{la résolution de ses dépendances}, voir \cite{kraken_entropy}.}\\
\bigbreak
\section{Conclusion}
Notre distribution atteint désormais un stade fonctionnel complet, incluant un gestionnaire de paquets \textbf{basique}.\\
La prochaine phase  produire une ISO installable pour permettre son déploiement massif.
\clearpage

                                        %----------------------- bootable iso  --------------------------------------------------

\chapter{  Création de l'ISO : Génération de l'image installable}
\minitoc
\clearpage
%\section{Création de l'ISO : Génération de l'image installable}




\section{Introduction}

Afin de rendre le système accessible à l'utilisateur final, nous devons créer une image ISO amorçable que celui-ci pourra télécharger et installer sur sa machine, que ce soit en environnement virtuel ou sur du matériel physique.

La création d’une image ISO est une tâche souvent complexe, tout comme la construction du système lui-même.

\begin{figure}[H]
  \centering
  \includegraphics[width=1\textwidth, height=10cm]{images_pfe/bootloader process.png}
  \caption{Processus de démarrage de Kraken OS}
  \label{fig:kbootproc}
\end{figure}

Lors de l’allumage du PC, un signal électrique déclenche le \textbf{POST} (\emph{Power-On Self-Test}), qui initialise le périphérique de démarrage (BIOS ou UEFI).\\

Ensuite, le BIOS/UEFI charge le chargeur d’amorçage (Syslinux) en lisant le fichier de configuration \texttt{isolinux.cfg}, ce qui permet de charger le noyau Linux (\texttt{vmlinuz-kraken}) ainsi que le disque RAM initial (\texttt{initrd}).

Ainsi, pour créer l'ISO, nous devons suivre les étapes suivantes~:
\begin{itemize}
    \item Générer le fichier SquashFS contenant notre système de fichiers
    \item Développer l'image \texttt{initrd}
    \item Configurer le chargeur d'amorçage Syslinux
    \item Développer un script et une application graphique d'installation pour les utilisateurs
    \item Générer le fichier ISO amorçable
\end{itemize}


\section{Générer le fichier SquashFS}

Depuis le chapitre \ref{chap:corebuild}, nous avons construit notre système sur un disque dur, précisément sur un disque VDI de VirtualBox. Ainsi, pour générer l'ISO, nous devons~:
\begin{itemize}
    \item Convertir ce disque VDI en fichier img au format RAW
    \item Monter le fichier (qui contient déjà plusieurs partitions) à l'aide d'outils comme \texttt{kpartx}
    
\end{itemize}

Nous devons ensuite~:
\begin{itemize}
    \item Nettoyer les fichiers de la partition montée (suppression des fichiers temporaires)
    \item Modifier les configurations système (Exemple  le fichier \texttt{/etc/fstab})
    
\end{itemize}

Enfin, nous générons le fichier SquashFS à l'aide de l'outil \texttt{mksquashfs}. Ce fichier contiendra l'intégralité du système dans un format compressé. Nous utilisons l'algorithme de compression \texttt{xz} pour optimiser l'espace.

\begin{figure}[H]
  \centering
  \includegraphics[width=1\textwidth]{images_pfe/rootfsfile.png}
  \caption{Fichier SquashFS généré}
  \label{fig:rootfs}
\end{figure}



\section{Image initrd}

Nous devons maintenant générer l’image \texttt{initrd}.  
Cette image est développée en utilisant \texttt{BusyBox}, une suite logicielle regroupant plusieurs utilitaires Unix dans un seul fichier exécutable.

\noindent
Nous avons besoin que \texttt{BusyBox} prenne en charge les boucles (\texttt{loop}), \texttt{SquashFS} et d’autres fonctionnalités nécessaires.

\begin{figure}[H]
  \centering
  \includegraphics[width=1\textwidth, height=10cm]{images_pfe/busyboxbinary.png}
  \caption{Structure de l'exécutable BusyBox}
  \label{fig:busybox}
\end{figure}

Cet \texttt{initrd} contient un script d’initialisation nommé \texttt{init}, essentiel au montage du fichier \texttt{SquashFS}. Ce script est écrit en \texttt{Bash} et fonctionne exclusivement sur \texttt{Kraken OS}.

\vspace{0.5cm}
\noindent
\textbf{Exemple du contenu du script \texttt{init} :}
\begin{figure}[H]
  \centering
  \includegraphics[width=1\textwidth, height=10cm]{images_pfe/intirdinitscript.png}
  \caption{Exemple de script \texttt{init}}
  \label{fig:initscript}
\end{figure}

\vspace{0.3cm}
\noindent
\textbf{Explication :}

L’objectif principal de ce script est de :
\begin{enumerate}
  \item Monter les systèmes de fichiers virtuels (\texttt{proc}, \texttt{dev}, \texttt{sys}, etc.) ;
  \item Monter le contenu du fichier ISO dans \texttt{/mnt/iso}, qui contient notre fichier \texttt{squashfs} ;
  \item Monter le fichier \texttt{squashfs}, contenant notre système de fichiers, dans \texttt{/mnt/live} ;
  \item Copier le contenu de \texttt{squashfs} depuis \texttt{/mnt/live} vers \texttt{/mnt/rootfs} ;
  \item Entrer dans l’environnement \texttt{chroot} situé dans \texttt{/mnt/rootfs} à l’aide de la commande \texttt{switch\_root}.
\end{enumerate}

\vspace{0.5cm}
\noindent
\textbf{Remarque :}

Vous vous demandez peut-être pourquoi nous montons le contenu du fichier \texttt{squashfs} à l’étape 3, puis en copions le contenu à l’étape 4.

C’est parce que la philosophie du système de fichiers \texttt{SquashFS} repose sur le principe du \textbf{mode lecture seule}.  
Pour pouvoir effectuer un \texttt{chroot} dans ce système, il aurait fallu monter le fichier \texttt{squashfs} via un système de fichiers de type \texttt{overlay}.

Cela nécessiterait :
\begin{itemize}
  \item De compiler notre noyau avec le support d’\texttt{overlayfs} ;
  \item De compiler \texttt{BusyBox} avec le support du type \texttt{overlay}.
\end{itemize}

Malheureusement, nous n’avons pas eu le temps de mettre en œuvre cette solution.

Nous avons donc choisi une autre approche :  
Monter le fichier \texttt{squashfs} via un \texttt{bind} dans un répertoire, puis copier son contenu dans un autre répertoire accessible en écriture.  
Ce n’est pas la méthode la plus optimale, mais elle est fonctionnelle pour cette version.  
Nous prévoyons d’améliorer cette partie dans une prochaine version de l’image ISO.

\vspace{0.5cm}
\noindent
Après cela, nous générons le fichier \texttt{initrd} à l’aide des outils \texttt{find}, \texttt{cpio}, \texttt{gzip}, etc.

\begin{figure}[H]
  \centering
  \includegraphics[width=1\textwidth, height=10cm]{images_pfe/initrdcontentfile.png}
  \caption{Contenu du fichier \texttt{initrd}}
  \label{fig:initcontent}
\end{figure}

\noindent
\textcolor{blue}{Le code source du fichier \texttt{initrd} et du script \texttt{init} est disponible dans \cite{initrd}.}


\section{Bootloader Syslinux}

Dans l'environnement ISO, il est recommandé d'utiliser le chargeur de démarrage Syslinux au lieu de GRUB pour des raisons de simplicité. Ensuite, pendant l'installation du système depuis l'environnement live vers le disque désiré, nous utiliserons le chargeur de démarrage GRUB. 

Il y a une configuration critique à effectuer dans Syslinux pour garantir un démarrage correct.\\
Exemple du fichier \texttt{isolinux.cfg} :

\begin{figure}[H]
  \centering
  \includegraphics[width=1\textwidth, height=10cm]{images_pfe/cfgisolinuxcorrect.png}
  \caption{Exemple fichier isolinux.cfg }
  \label{fig:isolinux}
\end{figure}

Exemple de menu de syslinux bootloader :

\begin{figure}[H]
  \centering
  \includegraphics[width=1\textwidth, height=10cm]{images_pfe/syslinuxmenu.png}
  \caption{Menu de démarrage Syslinux}
  \label{fig:syslinux}
\end{figure}
\textcolor{blue}{Pour en savoir plus sur syslinux, référez-vous à \cite{syslinux}.}\\
\section{Script d'installation }
\label{secc:sctipttui}

Après avoir démarré l'ISO , nous devons fournir à l'utilisateur un script capable d'installer l'ISO depuis \textbf{l'environnement live} vers le \textbf{disque dur local}.

Ce script gère plusieurs aspects, tels que la partition du disque, le montage du disque, la copie des fichiers système, la configuration du chargeur de démarrage dans le nouveau système, la création d'un utilisateur, la configuration de la langue, de la disposition du clavier, la configuration de l'environnement du nouvel utilisateur, l'édition du fichier \texttt{/etc/fstab}, l'activation de thèmes pour GRUB, SDDM, etc.

Ce script est développé spécialement pour Kraken OS ; si vous tentez de l'exécuter sur une autre machine ou distribution, il échouera.




\textbf{Commande :}

\begin{lstlisting}
./kraken_install.sh "disk_name" "home_on" "swap_on" "username" "userpass" 
  "system_language" "keyboard_layout" "hostname" "time_zone"
\end{lstlisting}

\textbf{Exemple d'utilisation :}
\begin{lstlisting}
sudo ./kraken_install.sh /dev/sda yes yes n1cef password en_US.UTF-8 us kraken /Africa/Tunis
\end{lstlisting}



\textbf{Exemple de parametre valide }
\begin{table}[H]
    \centering
    \small
    \begin{tabularx}{\textwidth}{|c|c|X|}
        \hline
        \textbf{Paramètre} & \textbf{Exemples} & \textbf{Description} \\
        \hline
        \texttt{disk-name} & \texttt{/dev/sda}, \texttt{/dev/sdb} & Chemin complet vers le disque cible \\
        \hline
        \texttt{home-on} & \texttt{yes}, \texttt{no} & Créer une partition séparée pour \texttt{/home} \\
        \hline
        \texttt{swap-on} & \texttt{yes}, \texttt{no} & Créer une partition de swap \\
        \hline
        \texttt{username} & \texttt{n1cef} & Nom du compte utilisateur principal  \\
        \hline
        \texttt{userpass} & \texttt{"MotDePasse123!"} & Mot de passe de l'utilisateur principal  \\
        \hline
        \texttt{system-language} & \texttt{fr-FR.UTF-8}, \texttt{en-US.UTF-8}, \texttt{ar-SA.UTF-8} & Langue et paramètre régional du système \\
        \hline
        \texttt{keyboard-layout} & \texttt{fr}, \texttt{us}, \texttt{ar} & Code de disposition du clavier (Français, Anglais, Arabe) \\
        \hline
        \texttt{hostname} & \texttt{kraken} & Nom d'hôte du système  \\
        \hline
        \texttt{time-zone} & \texttt{/Europe/Paris}, \texttt{/Africa/Tunis}, \texttt{/Asia/Dubai} & Fuseau horaire au format \texttt{/Région/Ville} \\
        \hline
    \end{tabularx}
    \caption{Paramètres de configuration système}
    \label{tab:system-config-params}
\end{table}


%\begin{figure}[H]
 % \centering
 % \includegraphics[width=1\textwidth, height=14cm]{images_pfe/diskuses.png}
 % \caption{ exemple d utilisation de script d’installation }
 % \label{fig:diskuses}
%\end{figure}



\textcolor{blue}{Le code source du script d’installation est disponible dans  \cite{installateur_tui}.}


\section{Application graphique "Kraken Installer" }
\label{secc:graphapp}


Le script d'installation Kraken créé précédemment peut être difficile à utiliser pour un utilisateur Linux non expert. Ainsi, nous avons choisi de développer une application graphique pour rendre le processus d'installation du système plus convivial.




Cette application fournit simplement à l'utilisateur des champs et des options pour sélectionner la langue du système, configurer son nom d'utilisateur et son mot de passe, choisir la disposition du clavier, sélectionner le disque et la partition, choisir les paquets à installer dans le nouveau système, etc.

\textbf{Fonctionnalité clé} : Génération d'un fichier JSON stockant toutes ces configurations pour l'installation ultérieure du système.


\begin{figure}[H]
  \centering
  \includegraphics[width=1\textwidth, height=10cm]{images_pfe/jsonsettings.png}
  \caption{Exemple de fichier settings.json généré}
  \label{fig:jsonsettings}
\end{figure}


\clearpage
\subsection*{Langages et frameworks utilisés}
\begin{figure}[hbt!]
  \centering
  \begin{minipage}[b]{0.18\textwidth}
    \includegraphics[width=\textwidth]{images_pfe/c+++.png}
    \caption{C++17/20}
  \end{minipage}\hfill
  \begin{minipage}[b]{0.18\textwidth}
    \includegraphics[width=\textwidth]{images_pfe/xml.png}
    \caption{Qt XML}
  \end{minipage}\hfill
  \begin{minipage}[b]{0.18\textwidth}
    \includegraphics[width=\textwidth]{images_pfe/qt6.jpeg}
    \caption{Qt6 Framework}
  \end{minipage}\hfill
  \begin{minipage}[b]{0.18\textwidth}
    \includegraphics[width=\textwidth]{images_pfe/css.jpeg}
    \caption{Qt CSS}
  \end{minipage}\hfill
  \caption{Stack technique de l'installateur graphique}
  \label{fig:krakenistallerlanguage}
\end{figure}
\FloatBarrier
 
  
  
  
 


\subsection{ Les pages de l'application}


\subsection{ Page principale}
\begin{figure}[H]
  \centering
  \includegraphics[width=1\textwidth, height=10cm]{images_pfe/mainwindow.png}
  \caption{Page principale de l'application Kraken Installer (page d'accueil)}
  \label{fig:mwindow}
\end{figure}



%\subsection{ Page de bienvenue}

%\begin{figure}[H]
 % \centering
 % \includegraphics[width=1\textwidth, height=10cm]{images_pfe/welcome.png}
 % \caption{Page de bienvenue de l'application Kraken Installer}
 % \label{fig:wlecompage}
%\end{figure}




\subsection{Page de configuration du clavier}
\begin{figure}[H]
  \centering
  \includegraphics[width=1\textwidth, height=10cm]{images_pfe/keyboard.png}
  \caption{Interface de configuration clavier de l'application \textsc{Kraken Installer}}
  \label{fig:keyboardpage}
\end{figure}

\textbf{Fonctionnalité} : Cette page propose à l'utilisateur :
\begin{itemize}
    \item Deux menus déroulants (\texttt{QComboBox}) pour :
    \begin{itemize}
        \item Sélectionner la langue du système
        \item Choisir la disposition du clavier
    \end{itemize}
    \item Un bouton de validation qui :
    \begin{itemize}
        \item Sauvegarde les paramètres dans un fichier \texttt{JSON}
        \item Redirige vers la page suivante
    \end{itemize}
\end{itemize}



\subsection{ Page de configuration de la localisation}

\begin{figure}[H]
  \centering
  \includegraphics[width=1\textwidth, height=10cm]{images_pfe/location.png}
  \caption{Interface de configuration géographique de l'application \textsc{Kraken Installer}}
  \label{fig:locationpage}
\end{figure}

\textbf{Fonctionnalité} : Cette page permet à l'utilisateur de :
\begin{itemize}
    \item Sélectionner sa région géographique via un \texttt{QListWidget}
    \item Choisir son fuseau horaire (ex: \texttt{Africa/Tunis}) via un \texttt{QComboBox}
    \item Valider ces paramètres avec un bouton \textit{Suivant} qui :
    \begin{itemize}
        \item Sauvegarde les configurations dans le fichier \texttt{JSON}
        \item Redirige vers l'étape suivante de l'installation
    \end{itemize}
\end{itemize}



\subsection{Page de configuration de l'utilisateur}

\begin{figure}[H]
  \centering
  \includegraphics[width=1\textwidth, height=10cm]{images_pfe/user.png}
  \caption{Interface de configuration utilisateur de l'application \textsc{Kraken Installer}}
  \label{fig:userpage}
\end{figure}


\textbf{Fonctionnalité} : Cette interface permet à l'utilisateur de :
\begin{itemize}
    \item Définir le nom d'hôte du système via un \texttt{QLineEdit}
    \item Créer un compte utilisateur avec :
    \begin{itemize}
        \item Un champ \texttt{QLineEdit} pour le nom d'utilisateur
        \item Deux champs de mot de passe (saisie + confirmation)
    \end{itemize}
    \item Valider via un bouton \textit{Suivant} qui :
    \begin{itemize}
        \item Vérifie la concordance des mots de passe
        \item Signale les champs incomplets
        \item Sauvegarde les paramètres dans le fichier \texttt{JSON}
        \item Redirige vers l'étape suivante
    \end{itemize}
\end{itemize}

\textbf{Note de sécurité} : 
\begin{itemize}
    \item Les mots de passe doivent être identiques
    \item Tous les champs sont obligatoires
    \item En cas d'erreur, l'application bloque la progression et demande une correction
\end{itemize}



\subsection{Page de partitionnement du disque}

\begin{figure}[H]
  \centering
  \includegraphics[width=1\textwidth, height=10cm]{images_pfe/disk.png}
  \caption{Interface de partitionnement de disque de l'application \textsc{Kraken Installer}}
  \label{fig:diskpage}
\end{figure}

\label{subsec:disk-partitioning}

\textbf{Fonctionnalités} :
\begin{itemize}
    \item \texttt{QTreeWidget} détectant automatiquement les disques disponibles
    \item Bouton \textit{Actualiser les disques} (\texttt{QPushButton}) pour mettre à jour la liste des disques 
    \item Options de partitionnement via des cases à cocher (\texttt{QCheckBox}) :
    \begin{itemize}
        \item Créer une partition \texttt{/home} séparée
        \item Activer une partition \texttt{swap}
        \item Case obligatoire :  Je confirme avoir compris que toutes les données du disque seront effacées
    \end{itemize}
    \item Validation via bouton \textit{Suivant} qui :
    \begin{itemize}
        \item Vérifie la sélection de la case de confirmation
        \item Sauvegarde la configuration dans le fichier \texttt{JSON}
        \item Redirige vers l'étape suivante
    \end{itemize}
\end{itemize}

\


\subsection{Page de sélection des paquets}



\label{subsec:package-selection}
\begin{figure}[H]
  \centering
  \includegraphics[width=1\textwidth, height=10cm]{images_pfe/packages.png}
  \caption{Interface de sélection des paquets de l'application \textsc{Kraken Installer}}
  \label{fig:packagepage}
\end{figure}



\textbf{Fonctionnalités} :  
Cette interface propose à l'utilisateur :
\begin{itemize}
    \item 3 onglets thématiques via \texttt{QTabWidget} :
    \begin{itemize}
        \item Informatique
        \item Mathématiques
        \item Physique
    \end{itemize}
    
    \item Sous-catégories détaillées (exemple pour l'onglet \textit{Informatique}) :
    \begin{itemize}
        \item Outils de développement général
        \item Outils web
        \item DevOps \& Cloud
        \item Développement mobile
        \item Cybersécurité
        \item IA/ML (Intelligence Artificielle et Machine Learning)
    \end{itemize}
    
    \item Sélection via \texttt{QCheckBox} pour chaque paquet
    \item Validation via \texttt{QPushButton} qui :
    \begin{itemize}
        \item Sauvegarde les sélections dans un tableau JSON
        \item Redirige vers l'étape suivante
    \end{itemize}
\end{itemize}





\subsection{Page de progression de l'installation}
\label{subsec:installation-progress}

\begin{figure}[H]
  \centering
  \includegraphics[width=1\textwidth, height=10cm]{images_pfe/installation.png}
  \caption{Interface de progression de l'installation dans \textsc{Kraken Installer}}
  \label{fig:install-page}
\end{figure}

\textbf{Fonctionnalités} :
\begin{itemize}
    \item Bouton \textit{Démarrer l'installation} (\texttt{QPushButton}) lançant le processus :
    \begin{itemize}
        \item Extraction des paramètres  depuis le fichier \texttt{JSON}
        \item Exécution d'un script dinstallation  \textcolor{blue}{\ref{secc:sctipttui}}   avec les paramètres en entrée
    \end{itemize}
    
    \item Widget \texttt{QPlainTextEdit} affichant :
    \begin{itemize}
        \item Les logs d'installation en temps réel
        \item La progression des étapes (partitionnement, copie des fichiers, configuration...)
        
    \end{itemize}
    
    \item Gestion visuelle de la progression :
    \begin{itemize}
        \item Barre de progression intégrée 
       
    \end{itemize}
\end{itemize}




\subsection{Page de redémarrage}


\begin{figure}[H]
  \centering
  \includegraphics[width=1\textwidth, height=10cm]{images_pfe/finish.png}
  \caption{Interface de confirmation de redémarrage dans \textsc{Kraken Installer}}
  \label{fig:reboot-page}
\end{figure}
\label{subsec:reboot}

\textbf{Fonctionnalité finale} :
\begin{itemize}
    \item Bouton de confirmation (\texttt{QPushButton}) demandant à l'utilisateur :
    \begin{itemize}
        \item De valider le redémarrage du système
        
    \end{itemize}
 
 \end{itemize}




\textcolor{blue}{Le code source du  lapplication  graphique d’installation est disponible dans  \cite{installateur_gui}.}
\subsection{ISO}

Maintenant, après avoir préparé tous les composants, on peut générer l'image ISO.\


\medskip
\clearpage
\textbf{Exemple de structure d'un fichier ISO :}

\begin{figure}[H]
  \centering
  \includegraphics[width=1\textwidth, height=10cm]{images_pfe/isocontent.png}
  \caption{Contenu du fichier ISO avant sa génération}
  \label{fig:iso_directory_content}
\end{figure}

\medskip

\textbf{Explication :}\\
Comme le montre la figure, à la racine du répertoire, nous avons le fichier \texttt{squashfs} généré précédemment ainsi qu'un répertoire nommé \texttt{boot}.\\
Dans ce répertoire \texttt{boot}, on trouve les fichiers \texttt{initrd.img}, l’image du noyau, les fichiers \texttt{config} et \texttt{System.map}, ainsi qu’un dossier \texttt{isolinux} contenant notre chargeur d’amorçage (SYSLINUX).\\
Ce dossier \texttt{isolinux} contient des fichiers critiques comme \texttt{isolinux.cfg} \textcolor{blue}{(créé précédemment \ref{fig:isolinux})}, ainsi que plusieurs bibliothèques et modules d’amorçage tels que \texttt{poweroff.c32}, \texttt{reboot.c32}, \texttt{libutil.c32} et \texttt{libmenu.c32}.

\medskip
\bigbreak

\textbf{Exemple de commande pour générer une ISO amorçable :}

\textbf{Remarque :} les grandes distributions créent généralement leurs propres outils pour la génération d'images ISO.\\
Par exemple, la distribution Arch Linux utilise son propre outil appelé \texttt{archiso}, tandis que la distribution Red Hat utilise \texttt{lorax}.\\
Malheureusement, nous n'avons pas encore eu le temps de développer notre propre outil, mais nous envisageons de le faire bientôt.\\
À ce stade, nous utilisons des outils comme \texttt{xorriso} et \texttt{mkisofs} pour générer l'ISO.
\begin{verbatim}
sudo mkisofs \
  -V "KRAKEN-OS" \
  -o /home/nacef/krakenos-3.10.iso \
  -b boot/isolinux/isolinux.bin \
  -c boot/isolinux/boot.cat \
  -no-emul-boot \
  iso-pfe-off
\end{verbatim}

\textbf{Explication des options :}
\begin{itemize}
  \item \texttt{-V "KRAKEN-OS"} : définit l'étiquette de l'ISO
  \item \texttt{-o} : chemin de sortie du fichier ISO
  \item \texttt{-b} : image de démarrage de isolinux
  \item \texttt{-c} : fichier catalogue de démarrage de isolinux
  \item \texttt{-no-emul-boot} : requis pour rendre l’image amorçable
  \item \texttt{iso-pfe-off} : répertoire contenant la structure du système de fichiers ISO
\end{itemize}

\medskip

\textbf{Exemple : détection du type, de la taille et de l’intégrité de l’image ISO à l’aide des commandes \texttt{file}, \texttt{du}, et \texttt{md5sum} :}

\begin{figure}[H]
  \centering
  \includegraphics[width=1\textwidth, height=10cm]{images_pfe/fileisoofff.png}
  \caption{Type du fichier ISO, taille et vérification d'intégrité}
  \label{fig:iso_type}
\end{figure}

\medskip
\textbf{Résumé :}\\
Lorsque l'ISO est démarrée (ex: machine virtuelle), le BIOS/UEFI charge le bootloader ISOLINUX (dans \texttt{isolinux/}), lit le fichier \texttt{isolinux.cfg}, affiche le menu, puis charge le noyau et l'initrd spécifiés dans l'entrée sélectionnée.\\

\texttt{initrd.img} exécute alors le script \texttt{init} \textcolor{blue}{(créé précédemment \ref{fig:initscript})} via \texttt{busybox}. Ce script :\\
1. Monte les systèmes de fichiers virtuels (\texttt{/proc}, \texttt{/sys}, \texttt{/dev})\\
2. Détecte et monte le contenu de l'ISO\\
3. Monte le système SquashFS (\texttt{root.sfs})\\
4. Effectue un \texttt{chroot} dans ce système.\\

Le contrôle est alors transféré à \texttt{System V}, qui démarre l'environnement KDE Plasma et lance l'application graphique d'installation \textcolor{blue}{(section \ref{secc:graphapp})}, permettant d'installer le système sur disque.

\bigbreak
Le lien de téléchargement de l’image ISO est disponible ici : \cite{iso_link}.

Des étapes supplémentaires pour le téléchargement et un guide d’utilisation de notre distribution  sont  disponibles ici :\cite{guide_iso}.




\section{Conclusion}
Ce processus de développement rigoureux a permis de maîtriser l'ensemble de la chaîne de création d'une distribution GNU/Linux :
TBD:
 



