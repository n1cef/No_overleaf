\chapter*{Remerciements}

%\section*{Remerciements}

Avant toute chose, je tiens à exprimer ma profonde gratitude à mon encadrant, \textbf{M. Hedi Tmar}, pour son accompagnement constant, sa disponibilité, sa patience, ainsi que la qualité de ses conseils tout au long de ce travail.

\bigbreak

Je remercie également avec une sincère reconnaissance \textbf{M. Fares Ayadi}, dont le soutien m’a permis d’intégrer l’équipe de Kusa pour effectuer ce stage de fin d’études. Son aide a joué un rôle déterminant dans la concrétisation de ce projet.

\bigbreak

J’adresse enfin mes remerciements aux membres du jury, pour l’honneur qu’ils me font en acceptant d’évaluer ce travail. 

\bigbreak
Je tiens à adresser mes plus sincères remerciements à l’équipe de \textbf{Linux From Scratch (LFS)}~\cite{lfs} pour leur travail remarquable et leur engagement envers la communauté du logiciel libre.\\
Leur documentation (le \textbf{LFS Book}~\cite{lfs_book}) a constitué une véritable pierre angulaire de mon projet, en particulier les premières étapes de compilation croisée et de construction de la chaîne d’outils (voir notre chapitre  \ref{chap:corebuild}) .
\bigbreak
Je souhaite également remercier \textbf{Greg Kroah-Hartman} pour son excellent livre \textit{Linux Kernel in a Nutshell}, que nous avons utilisé dans le chapitre \ref{sec:kernel-boot}, en particulier les chapitres 1 à 6 de son livre. Pour plus de détails , voir la référence~\cite{linux_nutshell}.
\bigbreak
Enfin, je tiens à remercier l’équipe de \textbf{Beyond Linux From Scratch (BLFS)} \cite{blfs} pour nous avoir fourni la base des \textit{bootscripts} (System V), que nous avons enrichie avec nos simple  scripts personnalisés. Leur documentation nous a également servi de référence pour de nombreux paquets que nous avons compilés depuis les sources, en complément d’autres sources telles que les sites officiels, les dépôts GitHub, etc.













\clearpage