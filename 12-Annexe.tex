\begin{appendix}
\chapter{annexe}


%\chapter*{Compilation de software depuit le source} \label{annex:config} 
\section{Configuration des packages} \label{sec:config-packages}
\medskip
    \noindent\textbf{a) Exemple de configuration de GCC (Autotools)}  
    \begin{verbatim}
../configure \
  --prefix=/usr \
  --disable-multilib \
  --with-system-zlib \
  --enable-default-pie \
  --enable-default-ssp \
  --enable-host-pie \
  --disable-fixincludes \
  --enable-languages=c,c++,fortran,go,objc,obj-c++,m2
    \end{verbatim}
    \textbf{Explications :}  \\
    \texttt{--prefix} : répertoire d’installation  \\
    \texttt{--disable}/\texttt{--enable} : (dés)activer des fonctionnalités  \\
    \texttt{--with} : préciser un chemin ou une option système \\

    \medskip
    \noindent\textbf{b) Exemple de configuration de LLVM (CMake + Ninja)}  
    \begin{verbatim}
CC=gcc CXX=g++ \
cmake -G Ninja .. \
  -D CMAKE_INSTALL_PREFIX=/usr \
  -D CMAKE_SKIP_INSTALL_RPATH=ON \
  -D LLVM_ENABLE_FFI=ON \
  -D CMAKE_BUILD_TYPE=Release \
  -D LLVM_BUILD_LLVM_DYLIB=ON \
  -D LLVM_LINK_LLVM_DYLIB=ON \
  -D LLVM_ENABLE_RTTI=ON \
  -D LLVM_TARGETS_TO_BUILD="host;AMDGPU" \
  -D LLVM_BINUTILS_INCDIR=/usr/include \
  -D LLVM_INCLUDE_BENCHMARKS=OFF \
  -D CLANG_DEFAULT_PIE_ON_LINUX=ON \
  -D CLANG_CONFIG_FILE_SYSTEM_DIR=/etc/clang
    \end{verbatim}

    \medskip
  
    \noindent\textbf{c) Exemple de configuration de GLib (Meson)}  
    \begin{verbatim}
meson setup builddir \
  --prefix=/usr \
  --buildtype=release \
  -D introspection=disabled \
  -D man=true
    \end{verbatim}
    \textbf{Explications :}  \\
    \texttt{--buildtype} : mode de compilation (debug/release)\\



%\chapter*{compilateur croise} \label{annex:crosscompiler} 
\section{le triplet système} \label{sec:triplet}
\medskip

Le champ \textit{kernel} (noyau) et le champ \textit{os} (système d'exploitation) proviennent historiquement 
d'un unique champ « système ». Une forme à trois champs reste valable aujourd'hui pour certains systèmes, 
par exemple \texttt{x86\_64-unknown-freebsd}.\\
Cependant, deux systèmes peuvent partager le même noyau tout 
en étant trop différents pour utiliser le même triplet. \\
Par exemple, Android sur un téléphone mobile est 
radicalement différent d'Ubuntu sur un serveur ARM64, bien qu'ils utilisent le même type de processeur 
(ARM64) et le même noyau (Linux).

Sans émulation, il est impossible d'exécuter un binaire serveur sur un mobile, ou inversement. Le champ 
« système » a donc été divisé en \textit{kernel} et \textit{os} pour lever les ambiguïtés. Dans nos exemples :
\begin{itemize}
    \item Android : \texttt{aarch64-unknown-linux-android}
    \item Ubuntu : \texttt{aarch64-unknown-linux-gnu}
\end{itemize}




\end{appendix}
